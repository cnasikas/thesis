%!TEX root = ../thesis.tex
\chapter{Zero Knowledge Proofs}
\label{zkp}

A proof of knowledge is a protocol that enables one party to convince another of the validity of a statement.
In a zero-knowledge proof, this is accomplished without revealing any information beyond the legitimacy of the proof~\cite{kiagias:crypto},
meaning that one party can prove to another party that a given statemnet is true, without conveying any information apart from
the fact that the statement is indeed true~\cite{wiki:zkp}.

Let $\calp$ be a prover and $\calv$ the verifier. $\calp$ must convince $\calv$ that she has some
knowledge of a statement $x$ without explicitly stating what she knows. We call this knowledge a witness $w$.
Both parties are aware of a predicate $R$ that will attest to $w$ being a valid witness to $x$~\cite{kiagias:crypto}.

\section{Examples}

\subsection{The strange cave of Ali Baba~\cite{Quisquater:1989:EZP:118209.118269}}

\begin{figure}[t!]
  \centering
  \begin{subfigure}[t]{0.30\textwidth}
    \centering
    \resizebox{\linewidth}{!}{
      \begin{tikzpicture}[scale=1]
        % Cave %
        \draw(0,5) -- (2,5) -- (2,8) -- (10,8) -- (10,4);
        \draw(0,4) --(2,4) -- (2,1) -- (10,1) -- (10, 4);
        \draw(9,4.5) -- (10,4.5);
        \draw(3,5) -- (3,7) -- (9,7) -- (9,5);
        \draw(3,4) -- (3,2) -- (9,2) -- (9,5);
        % Actors %
        \draw[fill] (1,5.6) circle [radius=0.1];
        \node [below] at (1,5.5) {Alice};
        \draw[fill] (2.5,4.6) circle [radius=0.1];
        \node [below] at (2.5,4.5) {Bob};
        % Arrows %
        \draw [dashed, ->] (2.5,4.6) -- (2.5,6);
        \draw [dashed, ->] (2.5,3.9) -- (2.5,2);

      \end{tikzpicture}
    }
    \caption{Alice stands outside and Bob choose randomly a path, left or right}
  \end{subfigure}
  \begin{subfigure}[t]{0.30\textwidth}
    \centering
    \resizebox{\linewidth}{!}{
      \begin{tikzpicture}[scale=1]
        % Cave %
        \draw(0,5) -- (2,5) -- (2,8) -- (10,8) -- (10,4);
        \draw(0,4) --(2,4) -- (2,1) -- (10,1) -- (10, 4);
        \draw(9,4.5) -- (10,4.5);
        \draw(3,5) -- (3,7) -- (9,7) -- (9,5);
        \draw(3,4) -- (3,2) -- (9,2) -- (9,5);
        % Actors %
        \draw[fill] (1,5.6) circle [radius=0.1];
        \node [below] at (1,5.5) {Alice};
        \draw[fill] (2.5,4.6) circle [radius=0.1];
        \node [below] at (2.5,4.5) {Bob};
        % Arrows %
        \draw [dashed, ->] (2.5,4.6) -- (2.5,6);
        \draw [dashed, ->] (2.5,3.9) -- (2.5,2);

      \end{tikzpicture}
    }
    \caption{Alice calls to Bob, asking him to come out either the left or the right passage}
  \end{subfigure}
  \begin{subfigure}[t]{0.30\textwidth}
    \centering
    \resizebox{\linewidth}{!}{
      \begin{tikzpicture}[scale=1]
        % Cave %
        \draw(0,5) -- (2,5) -- (2,8) -- (10,8) -- (10,4);
        \draw(0,4) --(2,4) -- (2,1) -- (10,1) -- (10, 4);
        \draw(9,4.5) -- (10,4.5);
        \draw(3,5) -- (3,7) -- (9,7) -- (9,5);
        \draw(3,4) -- (3,2) -- (9,2) -- (9,5);
        % Actors %
        \draw[fill] (1,5.6) circle [radius=0.1];
        \node [below] at (1,5.5) {Alice};
        \draw[fill] (2.5,4.6) circle [radius=0.1];
        \node [below] at (2.5,4.5) {Bob};
        % Arrows %
        \draw [dashed, ->] (2.5,4.6) -- (2.5,6);
        \draw [dashed, ->] (2.5,3.9) -- (2.5,2);

      \end{tikzpicture}
    }
    \caption{Bob complies appearing at the exit Alices names}
  \end{subfigure}
  \caption{Ali Baba Cave}
\end{figure}

\section{Formal Definition~\cite{kiagias:crypto}}

Let $<\calp, \calv>$ be a pair of interactive programs. Define $\text{out}^{\calp}_{\calp, \calv}(x,w,z)$
to be the output of $\calp$ when both $\calp$ and $\calv$ are executed with the public input $x$ and private
inputs $w$ and $z$ ($\calp$ determines $w$ and $\calv$ choose $z$); $\text{out}^{\calv}_{\calp, \calv}(x,w,z)$
is similar defined for $\calv$. The PPT interactive protocol $<\calp, \calv>$ is a \textbf{zero-knowledge proof}
for a language $L \in NP$ with knowledge error $k$ and zero-knowledge distance $\epsilon$ if the following
properties hold.

\begin{enumerate}
  \item \textbf{Completeness:}
  \item \textbf{Soundness:}
  \item \textbf{Zero-knowledge:}
\end{enumerate}
