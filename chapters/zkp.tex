%!TEX root = ../thesis.tex
\chapter{Zero Knowledge Proofs}
\label{zkp}

A proof of knowledge is a protocol that enables one party to convince another of the validity of a statement.
In a zero-knowledge proof, this is accomplished without revealing any information beyond the legitimacy of the proof\cite{kiagias:crypto},
meaning that one party can prove to another party that a given statemnet is true, without conveying any information apart from
the fact that the statement is indeed true\cite{wiki:zkp}. The one party is called the prover and the other one the verifier.

\subsection{Examples}

\begin{figure*}[t!]
  \centering
  \begin{tikzpicture}
    \draw(0,5) --(2,5) -- (6,8) -- (10,5);
    \draw(0,4) --(2,4) -- (6,1) -- (10,4);
    \draw(10,5) -- (10,4);
    \draw(9,4.5) -- (10,4.5);

    \draw(3,4.5) -- (6,7) -- (9,4.5);
    \draw(3,4.5) -- (6,2) -- (9,4.5);
  \end{tikzpicture}
\end{figure*}

\subsubsection{The strange cave of Ali Baba}
