%!TEX root = ../thesis.tex
\chapter{Introduction}
\label{introduction}

\section{Overview}
\label{introduction:overview}

In the big data era, huge amounts of data are constantly being collected and analyzed, evidently, leading innovation and
economic growth~\cite{10.1109/SPW.2015.27}. Data origins are various including wearable and IoT devices.
While data grows exponentially, gathering and storing them imposes high costs and liabilities.
Centralized organizations, amass large quantities of personal and sensitive information on which individuals have little or no control ~\cite{10.1109/SPW.2015.27}.
Furthermore, there is a growing public concern about user privacy.

Blockchain’s distributed nature eliminates the need for a trusted central authority and enables the connection between trustless entities.
The immutability and auditability of blockchain blocks and transactions, and cryptographical guarantees provided, can enforce liability, transparency and accountability without compromising privacy and security.
In addition, blockchain can significantly reduce data transaction costs between entities and increase transaction efficiency.
Like all technology, blockchain has limitations. It is argued that blockchain is not suited for high performance transactions or as a database replacement.
Due to the decentralized nature of blockchain and its necessity for a consensus mechanism the transaction rate remains quite low compared to financial services~\cite{Sompolinsky2015,Zohar:2015:BUH:2817191.2701411}.
Blockchain is not made for big data. The amount of data that blockchain can store and process is very limited so off-chain data frameworks are needed.
