%!TEX root = ../thesis.tex
\chapter{Introduction}
\label{introduction}

\section{Overview}
\label{introduction:overview}

In the Big Data area a huge amount of data are being created and gathered in a rapid pace~\cite{10.1109/SPW.2015.27}. Data origins are various including wearable and IoT devices. These data can provide valuable informations and aid in the analysis and the research in various sectors such as healthcare, machine learning, predicting earthquakes or weather phenomena and more. It is evident that those data can be beneficial for researchers and organizations. Many of these data are sensitive. For this reason, they are subject to specific privacy laws that make them inaccessible. However, the value of these data is big, as their importance to the various research, economic and other fields. Data processing in a privacy preserving manner is challenging.

The emergence of the blockchain in the end of 2008 highlighted new ways of data exchange without the need of a central authority to govern the action of the participants. For the first time in history, trustless entities can connect in a decentralized network and agree upon a global truth with security guarantees based on modern cryptography~\cite{10.1007/978-3-662-46803-6_10}. Blockchain has interesting properties which can make it a useful tool in the arsenal of data exchange and data processing. The immutability of the blockchain makes it a perfect candidate to serve as an irreversible log of all actions of the participants where they cannot deny.

Like all technology, blockchain has limitations. It is argued that blockchain is not suited for high performance transactions or as a database replacement. Due to the decentralized nature of blockchain and its necessity for a consensus mechanism the transaction rate remains quite low compared to financial services~\cite{Sompolinsky2015,Zohar:2015:BUH:2817191.2701411}.
Blockchain is not made for big data. The amount of data that blockchain can store and process is very limited and off-chain data frameworks must be combined.


\section{Thesis structure}
\label{introduction:structure}

The thesis is organized as follows. In Section~\ref{preliminaries}, we give common definitions and basic properties of various cryptographical primitives used in next sections. Section~\ref{blockchain} presents the blockchain, its history, and its internal components and how they work. In Section~\ref{problem} we analyze the various obstacles of privacy preserving data sharing and how the blockchain could be a useful component in the sharing ecosystem. Sections~\ref{solution} and~\ref{implemenation} provides the description of our solution, including security assumptions, high level architecture and implementation details. In Section~\ref{evaluation}, we present the results of various experiments regarding blockchain data storage and zkSNARK construction. Section~\ref{future_work}, presents various ideas that could improve our system, Section~\ref{related_work} discuss further related work, and finally Section~\ref{conclusion} concludes the thesis.
