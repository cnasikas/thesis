%!TEX root = ../thesis.tex
\chapter{Introduction}
\label{introduction}

\section{Overview}
\label{introduction:overview}

Some say we live in the Big Data era where data are being created and gathered in a rapid pace~\cite{10.1109/SPW.2015.27}. Technological breakthroughs over the past ten years in software and hardware have given rise to unprecedented ability to store and analyze data from various sources -- digital or, digitized, wearables or IoT devices with all kinds of sensors -- to personal data collected from everyday routine activities, medical records, bank accounts, mobility or CCTV. Such data are commonly used in statistical techniques which can identify patterns and relations that can  lead to the predictions of otherwise unknown events. They can, therefore, be beneficial for researchers and organizations, anyone who is interested in making sense of complex behavioral, financial, climatic, anthropological, and chemical phenomena. However, data related to people, personal data, are and should remain sensitive in terms of privacy. Many countries have forced correspond legislation, rendering such data legally inaccessible to anyone. The value of human data is huge, in financial, scientific and political terms. Harvesting and processing data in a privacy preserving manner is both very crucial and very challenging.

In 2008, the emergence of the blockchain technology highlighted new ways of data exchange. Its main property being its ability to achieve consensus among trustless entities connected through a decentralized network~\cite{10.1007/978-3-662-46803-6_10}, the blockchain could provide the cornerstone technology and logic for data sharing without compromising privacy.

The blockchain utilizes the cryptographic primitives necessary for transaction trackability and data provenance tracking. It is an immutable log that keeps record of all the participants's actions. It ensures the accountability of the participants consequently enforcing non-repudiation. Such properties of the blockchain make it a useful tool for data exchange and processing.

Choosing to use the blockchain to perform as a system controller means also having to deal with its limitations. It is argued that the blockchain is unsuitable for high performance transactions~\cite{Sompolinsky2015,Zohar:2015:BUH:2817191.2701411} or as a database replacement. The existing technology behind blockchain is not made for big data. The amount of data that blockchain can store and process is very limited and off-chain data frameworks needs to be combined.

The implementation of the above concepts, trying to overcome the recurring problems, is the main goal of this thesis. Alongside the previous goals, emphasis is put on zero knowledge verifiable computations which can produce a proof of correctness of computation over a dataset allowing the verification of the proof hiding sensitive input.


\section{Thesis structure}
\label{introduction:structure}

The present thesis is organized as follows. In Section~\ref{preliminaries}, we give common definitions and basic properties of various cryptographic primitives used in the following sections. In Section~\ref{blockchain} the blockchain is presented -- its history, its internal components, and how these components work. In Section~\ref{problem} we analyze the various obstacles of privacy preserving data sharing and how the blockchain could be a useful component in the sharing ecosystem. In Section~\ref{solution} and~\ref{implemenation} a solution is being developed and analyzed, including security assumptions, high level architecture and implementation details. In Section~\ref{evaluation}, we present the results of various experiments regarding blockchain data storage and zkSNARK construction. Various ideas for system improvement are presented in Section~\ref{future_work}. Further and related work is also discussed in Section~\ref{related_work}. Finally, a concluding statement summing up the findings of this research is included in Section~\ref{conclusion}.
