%!TEX root = ../thesis.tex
\chapter{Related Work}
\label{related_work}

\section{Enigma}
\label{rel_work:enigma}

Enigma is a blockchain-based protocol where decentralized applications can be deployed. It combines a blockchain with an off-chain storage where data are stored and enables different parties through a peer-to-peer network to run computation on them in a privacy preserving manner. Data are stored in a distributed hash-table (DHT) that is accessible through the blockchain. Data are not stored directly in the DHT as they are private. The DHT stores only references to the data that are encrypted and stored on the client side. The blockchain is used as an access-control manager and a log system which manages data access and identity verification. For computation an optimize verifiable secure MPC is used between computational nodes of the network where queries run, without a trusted third party.

\section{MedRec}
\label{rel_work:medrec}

MedRec is a decentralized record management system to handle Electronic Health Records (EHRs) with the use of the Ethereum blockhain. Its purpose is to address the various problems the health care industry faces especially when it comes to medical records exchange between various organizations. With the use of smart contracts a dynamic consent mechanism is created where patients form contracts with various providers that enforce them to follow the various access permissions the patient authorized. The medical records are stored on providers without any need to change their infrastructure. A gatekeeper node is installed in each provider and is responsible to listen to queries, that are predefined and crafted based on patient access permissions, and fulfill them. When a request for query execution arrives the provider checks the blockchain for access permissions and then it executes it locally and return the results.

\section{Datum}
\label{rel_work:datum}

Datum is a decentralized market place where users can share or sell their data. Data are encrypted and stored in a decentralized data store running by storing nodes. Storing nodes are paid by the users for their service in a token called DAT. Data, before leaving the client storage, are cleaned from personally identifying information and then saved to the network. Data consumers can request and buy datasets that are stored in the network. When a user accepts a buy request from a data consumer, it authorized it and the decryption key is sent to the data consumer which in turn pays the user with DAT. The procedure of data exchange is governed by a smart contract responsible for fairness in the system.

\section{ADSNARK}
\label{rel_work:adsnark}

Succinct Non-Interactive Arguments of Knowledge on Authenticated Data (AD-SNARK) is a cryptographic primitive in which privacy-preserving proofs on authenticated data can be generated and verified. In a AD-SNARK setting a party is requested for a proof of computation over a dataset acquired from a trusted source who vouch for the quality and legitimacy of the data. The verifier is able to check the validity of the dataset authenticated by the source. The creators of ADSNARK are inspired by ZQL~\cite{zql}, a query language for expressing simple computations on private data. ADSNARK is achieved by embedding digital signatures into SNARKs. The data source, which is trusted, authorizes the data by producing a valid digital signature which is taken as public input to the proof generation algorithm.
