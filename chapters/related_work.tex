%!TEX root = ../thesis.tex
\chapter{Related Work}
\label{related_work}

\section{Enigma}
\label{rel_work:enigma}

Enigma is a blockchain-based protocol where decentralized applications can be deployed. It combines a blockchain with an off-chain storage where data are stored and enables different parties through a peer-to-peer network to run computation on them in a privacy preserving manner. Data are stored in a distributed hash-table (or DHT) that is accessible through the blockchain. Data are not stored directly in the DHT as they are private. The DHT stores only references to the data that are encrypted and stored on the client side. The blockchain is used as an access-control manager and a log system which manages data access and identity verification. For computation an optimize verifiable secure MPC is used between computational nodes of the network where queries run, without a trusted third party.
