%!TEX root = ../thesis.tex

% abbreviations table
\abbreviations
\begin{center}
	\renewcommand{\arraystretch}{1.5}
	\begin{longtable}{ l @{\qquad} l }
	\toprule
	SET    & Secure Electronic Transaction \\
	SNARK & Succinct Non-Interactive Argument of Knowledge \\
	zkSNARK    & Zero-Knowledge Succinct Non-Interactive Argument of Knowledge \\
	UTXO    & Unspent transaction output \\
	SHA    & Secure Hash Algorithm \\
	PoW    & Proof of Work \\
	PoS    & Proof of Stake \\
	PBFT    & Practical Byzantine Fault Tolerance \\
	EVM    & Ethereum Virtual Machine \\
	ASIC    & Application-specific Integrated Circuit \\
	MPC    & Multi-party Computation \\
	DHT    & Distributed Hash Table \\
	Dapp    & Decentralised Application \\
	GDPR    & General Data Protection Regulation \\
	DPR    & Data Protection Directive \\
	EU    & European Union \\
	Opcode    & Operation Code \\
	VC    & Verifiable Computation \\
	CLI    & Command-line interface \\
	PKI    & Public Key Infrastructure \\
	CRS    & Common Reference String \\
	QSP    & Quadratic Span Program \\
	QAP    & Quadratic Arithmetic Program \\
	R1CS    & Rank 1 Constraint System \\
	URI    & Uniform Resource Identifier \\
	REST    & Representational State Transfer \\
	HTTP    & Hypertext Transfer Protocol \\
	UI    & User Interface \\
	SegWit    & Segregated Witness \\
	SPOF    & Single Point of Failure \\
	DoS    & Denial-of-service \\
	DDoS    & Distributed denial-of-service \\
	IoT    & Internet of Things \\
	DDH    & Decisional Diffie Hellman \\
	PPT    & Probabilistic polynomial-time \\
	PID    & Persistent identifiers \\
	ECC    & Elliptic curves \\
	OTP    & One-time pad\\
	PRG    & Pseudorandom generator\\
	ECDLP    & Elliptic Curve Discrete Logarithm Problem\\
	ECDSA    & Elliptic Curve Digital Signature Algorithm\\
	ECDH    & Elliptic Curve Diffie–Hellman\\
	DSA    & Digital Signature Algorithm\\
	NIST    & Digital Signature Algorithm\\
	DLGO    & Discrete logarithm\\
	CDH    & Computational Diffie-Hellman\\
	DDH    & Decisional Diffie-Hellman\\
	ECB    & Electronic Codebook\\
	CBC    & Cipher Block Chaining\\
	OFB    & Output Feedback\\
	CTR    & Coutner\\
	IV    & Initial Vector\\
	\bottomrule
	\end{longtable}
\end{center}
