%!TEX root = ../thesis.tex
\chapter{Blockchain}
\label{blockchain}

A blockchain is a distributed transaction ledger~\cite{nakamoto2012bitcoin}.
Blockchain consist of a continuously growing list of records, which are called blocks, and are linked, secured and immutable.
Each block typically contains a hash pointer being a link to a previous block, a timestamp and transaction data~\cite{wiki:blockchain}.
Blockchain has a high Byzantine fault tolerance because of a decentralized consensus mechanism.
This makes blockchains potentially suitable for fincancial activities, the recording of events, medical records~\cite{blockchain_ehr,Azaria2016}, and other records management activities,
such as identity management, transaction processing or documenting provenance~\cite{wiki:blockchain}. Blockchain could be seen as a distributed immutable, tamper-proof,
and audit log that records every data transaction. As a result any attempt to tamper blockchain is immediately evident and easily detectable.
The first successful financial implementation of blockchain is Bitcoin~\cite{nakamoto2012bitcoin}.
Bitcoin is the first decentralized digital currency and payment system invented by an unknown person or group of people under the name Satoshi Nakamoto~\cite{nakamoto2012bitcoin,wiki:bitcoin}.

\section{History}\label{blockchain:history}

Credit card transactions are the dominant payment method that is used on the web today~\cite{Narayanan:2016:BCT:2994437} handled by a financial
system involving processors, banks, credit card companies and other intermediaries. Normally, a credit card transaction is done as follow:
the buyer sends over his credit card details to the merchant, and then the merchant sends and validates the data in the financial system.

Buyers may feel uncofortable to handle their credit card details to an unknown vendor without good reputation, especially over a insecure channel.
Intermediate services, such as Paypal, sits between the buyer and the seller and the intermediate service has the buyer's credit card details which approves
the trasnaction and notify the seller. Through this architecture the buyer is not exposed to security risks and can be anonymous improving privacy. On the other hand,
both the buyer and the seller might have an account to the same service.

A notable intermediate architecture called SET (Secure Electronic Transaction) established in 1996 by VISA and MasterCard~\cite{wiki:set}
which goal was to combine the card associations' similar but incompatible protocols. SET allowed parties to identity themselves with the use of
certificates avoiding that way the need of having to enroll with the intermediary. SET failed to gain attraction in the market and the fundamental
problem has to do with the end-user certificates issue complexity.

David Chaum~\cite{wiki:david_chaum} is the inventor of the secure digital cash idea which first introduced in his 1983 paper~\cite{Chaum1983}. In
1990 proposed the first off-line e-cash system~\cite{Chaum:1988:UEC:646753.704915} and founded DigiCash~\cite{wiki:digicash}, an electronic money corporation,
which in 1994 sent the first electronic payment.

Digital cash shemes have a potential flaw called double-spending~\cite{wiki:double_spend} in which the same digital token can be spent more than once.
Chaum find a way to both keep the system anonymous and prevent double-spending with the use of blind signatures~\cite{Chaum1983,Chaum:1988:UEC:646753.704915}.
Nevertheless, Chauma's solution need a centralized trusted oracle that validates the transactions. Many cryptographers tried to improve Chaum's et al. sheme
such as Okamoto and Ohta~\cite{Watanabe1996} who implemented the subdivisions of coins with the use of Merkle trees~\cite{wiki:merkle_tree}.

Chaum patent the blind-signature scheme preventing others from developing ecash system that use the same protocol. A group of cryptographers called
Cypherpunk~\cite{wiki:cypherpunk} implemented an ecash system called MagicMoney which violated Chaum's patents.

Chaum's~\cite{wiki:david_chaum} ideas have been described as the technical roots of the vision of the Cypherpunk~\cite{wiki:cypherpunk}
movement that began in the late 1980s.

\section{Structure}\label{blockchain:structure}

\section{Identity}\label{blockchain:identity}

\section{Incentives}\label{blockchain:incentives}

For miners to participate in the network and contribute their computation resources incentives must be provided.
Normally, in permissionless blockchains such as Bitcoin or Ethereum a monetary incentive in the form of cryptocurrency
incentivize the miners and in a permissioned blockchain finance incentives or access to blockchain data could encourage miners to participate~\cite{deloitte}.

\section{Consensus mechanisms}\label{blockchain:consensus_mechanisms}

The consensus mechanism is the core mechanism of the blockchain. Through consensus, the shared state of the ledger comes to an agreement upon a global state,
allowing all the nodes of the network to reach the same ledger state. Achieving consensus in a distributed system is challenging.
A consensus mechanism has to be resilient to node failures, network delays and the existence of malicious nodes~\cite{wiki:byzantine_fault_tolerance}.

There are three basic consensus mechanism categories:

\begin{enumerate}
  \item Proof-of-Work (PoW).
  \item Proof-of-Stake (PoS).
  \item Practical Byzantine Fault Tolerance (PBFT)
\end{enumerate}


\subsection{Proof of Work (PoW)}\label{blockchain:consensus:pow}

In the case of Proof-of-Work consensus mechanism, each network node participating in
the consensus tries to solve a computational puzzle that is computational hard, but feasible to find, and easy to verify correctness.
A PoW system is also based on randomness, which is approximated proportionally to the computational power of each node.
The participating nodes are bound to trial and error until a correct answer is found. Being a blockchain system a random node is selected as a winner for block creation.
In order for a block to be accepted by the rest of consensus participants, they must complete a proof of work based on the data of the block.
The difficulty of the work is adjusted in a way that a new block is generated in a specific time interval on average.

The process of solving a PoW computational puzzle is known as mining and the participants as miners.
The first miner that solves the puzzle gets to add the proposed block to the blockchain and broadcast
the block to the entire network. Due to network latency and the distributed nature of the system it is possible that two miners will
find a block around the same time and some nodes will accept the first block and some others the second one. In such cases there is a temporary
fork in the blockchain, where one node is adding blocks to one branch while other nodes are adding blocks to another branch. Eventually the
system will come to an agreement, the longest branch will be accepted and the others will be discarded.

Bitcoin~\cite{Zohar:2015:BUH:2817191.2701411} is the first blockchain system that utilizes a PoW for blockchain block generation. Due to the nature of PoW, bitcoin is vulnerable to 51\% attacks.
This means that if a node or a group of nodes is able to control over 51\% of the computational power of the network, it is able to manipulate
the verification process: by writing its own blocks, it has made it possible to double-spend funds or reject certain transactions.

PoW consensus mechanism works very well in public blockchain systems where trust of the nodes is low eliminating
the double-spend problem -to redirect previously processed payments and use the same money twice- and guarding against Sybil attacks~\cite{Vu:2009:PCP:1671222}.
However, the transaction confirmation time is longer compared to conventional financial services (such as VISA)\cite{Sompolinsky2015,Zohar:2015:BUH:2817191.2701411,DBLP:journals/corr/abs-1708-05665} resulting
in slower transaction confirmation rates. Lastly, the energy waste attributed to the mining process can be very high -the energy requirements
of the Bitcoin protocol are estimated to be comparable to those of a small country~\cite{6912770}.

\subsection{Proof of Stake (PoS)}\label{blockchain:consensus:pos}

Proof-of-Stake algorithms are designed to overcome the disadvantages of PoW in terms of the high electricity consumption involved in mining~\cite{bl_consensus}
and provide equal security guarantees~\cite{Kiayias2017}. Unlike PoW where the miners solve computational puzzles in order to create a new block, in PoS the choice
of the block creator among the miners is random, yet relative to the stake the miner possesses according to the current blockchain ledger. Maintaining
the blockchain relies on the stakeholders themselves and assigns work to them based on the amount of stake that each possesses as reported in the ledge~\cite{Kiayias2017}.
The higher the stake participant, the higher the possibility to be chosen.

Proof-of-Stake algorithms suffer from the so-called “nothing at stake” problem.
The “nothing at stake” problem refers to attacks against PoS blockchain systems where shareholders do not have
incentives to follow the protocol and vote simultaneously on multiple blockchains exploiting the fact that little computational effort
is needed to build a PoS blockchain~\cite{Kiayias2017}. Ouroboros is the first provable secure PoS algorithm~\cite{Kiayias2017} and is the main consensus algorithm of the Cardano blockchain~\cite{cardano_site}.

\subsection{Practical Byzantine Fault Tolerance (PBFT)}\label{blockchain:consensus:PBFT}

The Practical Byzantine Fault Tolerance algorithm~\cite{Castro:1999:PBF:296806.296824} is the first practical consensus algorithm with Byzantine Fault Tolerance~\cite{wiki:byzantine_fault_tolerance}.
It is based on the concept of state machine replication and replication state voting, and is able to process tens of thousands of requests per second with minimal latency.
The algorithm has only a 3\% overhead over a typical filesystem~\cite{Castro:1999:PBF:296806.296824}.

PBFT and state-machine replication protocols’ downside is related to scalability, in terms of the number of nodes (replicas)~\cite{Vukolić2016} that can be supported.
PBFT has only been scaled and studied up to 20 replicas~\cite{bl_consensus,Vukolić2016}. To overcome this limitation, without compromising security, various PBFT variants,
such as Ripple and Stellar, partition the network into smaller groups called federates and each one runs a local consensus protocol among its members and
global consensus is achieved when certain conditions are being met~\cite{DBLP:journals/corr/abs-1708-05665}.

\begin{table}[]
  \centering
  \resizebox{\textwidth}{!}{
    \begin{tabular}{|l|l|l|l|}
      \hline
      & PoW &	PoS &	PBFT \\ \hline
      Blockchain Type &	Permissionless &	Both &	Permissioned \\ \hline
      Scalability of nodes &	High &	High &	Low \\ \hline
      Scalability of clients &	High &	High &	High \\ \hline
      Transaction rate &	Low &	High &	High \\ \hline
      Latency &	High &	Low &	Minimal \\ \hline
      Power consumption &	High &	Low &	Low \\ \hline
      Token needed &	Yes &	Yes &	No \\ \hline
      Cost of participation &	Yes &	Yes &	No \\ \hline
      Adversary tolerance &	<= 25\%	& Depends &	<= 33\% \\ \hline
    \end{tabular}
  }
  \caption{Blockchain consensus mechanisms. Adapted and modified from~\cite{bl_consensus,Vukolić2016}}
  \label{table:blockchain_consensus}
\end{table}

\section{Blockchain Types}\label{blockchain:blockchain_types}

There are various types of blockchains varying in restrictions on data access and participation in the consensus process.
Each one has its own advantages and disadvantages.

\begin{itemize}
  \item Public Blockchain: A public blockchain is a blockchain, in which there are no restrictions on reading blockchain data -encrypted or not- and validating transactions~\cite{prbc_vs_pubbc}.
  The most common implementation of a public blockchain is Bitcoin~\cite{nakamoto2012bitcoin} and Ethereum~\cite{ethash}.
  \item Federated or Consortium blockchain: In a federated blockchain transaction validation is limited to a predefined list of entities with their identities known to the network. Data access can either be public or restricted~\cite{prbc_vs_pubbc}.
  \item Private blockchain: A private blockchain is a blockchain where consensus mechanism is centralized to one single entity regardless of data access~\cite{prbc_vs_pubbc}.
\end{itemize}

\section{Consensus defined types of Blockchain}\label{blockchain:consensus_blockchain_types}

\begin{itemize}
  \item Permissionless blockchain: A permissionless blockchain is a blockchain, in which there are no restrictions on identities of transaction processors (i.e., users that are eligible to create blocks of transactions)~\cite{prbc_vs_pubbc}.
  \item Permissioned blockchain: A permissioned blockchain is a blockchain, in which transaction processing is performed by a predefined list of subjects with known identities~\cite{prbc_vs_pubbc}.
\end{itemize}

\begin{table}[]
  \centering
  \resizebox{\textwidth}{!}{
    \begin{tabular}{|l|l|l|l|}
      \hline
       & Public & Permissioned (Multiple Entities) & Private (Single Entity) \\ \hline
       Participants & Permissionless - Anonymous & Permissioned - Identified, Trusted & Permissioned - Identified, Trusted \\ \hline
       Data Access & Public & Public or Restricted & Restricted \\ \hline
       Consensus & PoW, PoS & FBTA, PoS & FBTA \\ \hline
    \end{tabular}
  }
  \caption{Blockchain Types. Source~\cite{hub-bl-types}}
  \label{table:blockchain_types}
\end{table}

\section{Blockchain Implementations}\label{blockchain_implementations}

\subsection{Bitcoin}\label{blockchain:impl:bitcoin}
\subsection{Ethereum}\label{blockchain:impl:ethereum}

Ethereum is an open-source, public, blockchain-based distributed computing platform featuring smart contract functionality.
It provides a decentralized Turing-complete virtual machine, the Ethereum Virtual Machine (EVM), which can execute scripts using an
international network of public nodes~\cite{wiki:ethereum}. Ethereum provides a cryptocurrency called ether which can be transferred between accounts and gas,
an internal pricing mechanism used to execute contracts and allocate resources on the network. Through EVM, Ethereum provides Solidity a
Turing-complete language for writing smart contracts that enables anyone to build distributed applications making the process much easier and efficient than ever before.
Ethereum use a Proof-of-Work consensus mechanism called Ethash and has been designed to be ASIC-resistant~\cite{ethash}.
Soon Ethereum will be moved to a Proof-of-Stake consensus mechanism called Casper. Ethereum can be implemented as well as a permissioned blockchain~\cite{consortium_chain_development,quorum}.


\subsection{Hyperledger}\label{blockchain:impl:hyperledger}

Hyperledger is an open-source permissioned blockchain and provides a flexible, modular and secure architecture with a
pluggable consensus mechanism is among the most popular permissioned blockchains~\cite{DBLP:journals/corr/abs-1708-05665}. In Hyperledger, Fabric a predefined list of
entities is not only known, but their identities and roles are registered and verified with a central registry service running within the system.
It also supports smart contracts on the blockchain, also known as chaincode~\cite{bl_consensus} and does not support a cryptocurrency system. The default consensus
mechanism of Hyperledger is the Practical Byzantine Fault Tolerance algorithm (PBFT) which assumes authenticated nodes.
Hyperledger version 0.6 fails to scale up to more than 16 nodes~\cite{DBLP:journals/corr/abs-1708-05665} but version 1.0 architecture has been designed to address transaction and
node scalability in a manner adherent with regulatory and industry standard~\cite{imb_hypeledger_adv}.

\subsection{Cardano}\label{blockchain:impl:cardano}

Cardano is a security focused blockchain that utilize the latest research and engineering insights to build a platform suitable for
the highest value applications~\cite{cardano_site}. It supports distributed applications creation and smart contracts verifiable by a method called formal
verification allowing logical proof of correctness of code providing high security. Cardano addresses the need for regulatory oversight while
maintaining consumer privacy and security. Cardano is the first blockchain project to be peer reviewed by academic researchers~\cite{cardano_site} and its consensus
mechanism, Ouroboros, is the first Proof of Stake algorithm to be provably secure~\cite{Kiayias2017}.
Cardano consists of two main layers, one for accounting and one for computation. The accounting layer is called Cardano Settlement Layer (CSL)
and the computation layer Cardano Computation Layer (CCP) where distributed application can be built and run upon. The CCP layer has not been
implemented yet and there is a plan to be released as a beta by the first quarter of 2018~\cite{cardano_parsons}.

\subsection{Enigma}\label{blockchain:impl:enigma}

Enigma is a decentralized computation platform with guaranteed privacy~\cite{DBLP:journals/corr/ZyskindNP15}. It operates through a peer-to-peer network enabling
different parties to store and run computation on data while keeping privacy. Enigma uses a highly optimized version of secure multi-party computation (MPC)
guaranteed by a verifiable secret-sharing scheme~\cite{DBLP:journals/corr/ZyskindNP15}. A blockchain is utilized as the controller of the network, managing access to the data and identities.
The data are stored off-chain, encrypted in a distributed database, and a modified distributed hash table (DTH), accessible through the blockchain, is used for holding only references to the data.
