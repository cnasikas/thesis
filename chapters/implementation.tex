%!TEX root = ../thesis.tex
\chapter{Implementation}
\label{implemenation}

To cover the needs of the data sharing and processing ecosystem the application is separated in two parts: the \textit{off-chain network} and the \textit{blockchain network}. The off-chain network is responsible for dataset storage, dataset distribution, and dataset processing. The blockchain is responsible for recording dataset registrations, requests for data processing and data processing outputs along with their proofs of computations. For that reason, every node of the network -- \textit{a data share node} -- is connected to the blockchain and track every transaction related to the application. The node can track and index every registered dataset, all requests for data processing and all processing outputs and proofs. A node, depending on its role, acts accordingly when a blockchain transaction is broadcasted in the network.

A variety of tools and APIs are made to implement the protocol as described in Chapter~\ref{solution}. They are analyzed in the following sections.

\section{API}

A server is implemented to aid developers and users and is responsible to: a) expose a RESTful API that facilitates the interaction with the application, and b) listen and record to a relational database all the events of the application emitted by the blockchain.

The existence of such server is not a threat to the decentralization of the system. The application is governed by the smart contracts deployed on the blockchain. Anyone can interact directly with the blockchain or implement their own server or client.

\subsection{RESTful API}
\label{implemenation:api:rest}

A RESTful API is provided to facilitate communication between any application, that follows the REST architecture, and the data sharing application. The REST API expose the blockchain business network that can be easily consumed by HTTP or REST clients. That way, any developer familiar with existing web technologies and frameworks is not obligated to learn the interval mechanisms of the blockchain system to develop and deploy applications atop.

Through the REST API one can:

\begin{itemize}
  \item Get all datasets
  \item Register a dataset (as pending)
  \item Get a specific dataset
  \item Get all processing requests
  \item Register a request for processing (as pending)
  \item Get a specific request
  \item Get all processors
  \item Register a processor (as pending)
  \item Get all controllers
  \item Register a controller (as pending)
  \item Get all blockchain accounts
\end{itemize}

The REST API expose a set of routes each one having an HTTP method and a URI. All available routes are shown at Table~\ref{table:api_routes}.

\begin{table}[ht!]
\centering
\caption{RESTful API Routes}
\begin{tabular}{|l|l|}
\hline
 Method & URI  \\ \hline
 GET & /\  \\ \hline
 GET &  /contracts \\ \hline
 GET &  /datastore \\ \hline
 GET &  /datastore/\{data\} \\ \hline
 POST &  /datastore\\ \hline
 GET &  /accounts \\ \hline
 GET &  /accounts/\{account\} \\ \hline
 GET &  /requests \\ \hline
 GET &  /requests/\{request\} \\ \hline
 POST &  /requests \\ \hline
 GET &  /processors \\ \hline
 POST &  /processors \\ \hline
 GET &  /controllers \\ \hline
 POST &  /controllers \\ \hline
\end{tabular}
\label{table:api_routes}
\end{table}

All blockchain transaction made by a user must be sign with her signing key. As keys should be private and be managed only by the user itself, the server exposing the REST API cannot make a transaction on behalf of the user. For that reason, all \verb|POST| actions create a database record where the status of the action is set as \verb|pending|.

\subsection{Database}
\label{implemenation:api:db}

The server is registered to all available events (§~\ref{implemenation:contracts:events}) of the system that are emitted by the blockchain. When an event is triggered a record of the action is saved to a relational database. The database's scheme is shown in Figure~\ref{fig:implemenation:db}.

\begin{figure}[ht!]
  \begin{tikzpicture}

    \matrix(Users) [table, label=above:Users] {
      id & firstName & lastName & email \\ & & & \\
    };

    \matrix(Addr) [table, label=above:Addresses, below=1.25cm of Users] {
      id & hash & userId \\ & & \\
    };

    \matrix(Datasets) [table, label=above:Datasets, below=1.25cm of Addr] {
      id & addressId & name & location & category & hash & txId & metaHash & status  \\ & & & & & & & & \\
    };

    \matrix(Requests) [table, label=above:Requests, below=1.25cm of Datasets] {
      id & datasetId & addressId & algorithmId & processed & pubKey & txId & status  \\ & & & & & & & \\
    };

    \matrix(Alg) [table, label=above:Algorithms, below=1.25cm of Requests] {
      id & name \\ & \\
    };

    \matrix(Processors) [table, label=above:Processors, below=1.25cm of Alg] {
      id & name & pubKey & addressId & txId & status \\ & & & & & \\
    };

    \matrix(Controllers) [table, label=above:Controllers, below=1.25cm of Processors] {
      id & name & pubKey & addressId & txId & status \\ & & & & & \\
    };

    \draw (Users-2-1.south)--++(0,-.5)-|(Addr-1-3.north);
    \draw (Addr-2-1.south)--++(0,-.5)-|(Datasets-1-2.north);

    \draw (Datasets-2-1.south)--++(0,-.5)-|(Requests-1-2.north);
    \draw (Requests-2-3.south)--++(0,-.5)-|(Alg-1-1.north);

  \end{tikzpicture}
  \caption{Database scheme}
  \label{fig:implemenation:db}
\end{figure}

\section{Distributed Application}
\label{implemenation:dapp}

The distributed application is a web-based application providing a graphical user interface (UI) to facilitate the data sharing platform usage. The goal of the distributed application is to provide to the end-user an easy, self-explanatory and efficient way of interacting with the data sharing ecosystem. As most users are already familiar with the use of web applications and platforms, such as webmails, cloud services and social media, such interfaces makes the user feel accustomed. The only requirement is to have an ethereum account and installed the metamask wallet~\cite{metamask}.

The distributed application consumes the REST API and yields the same functionalities.

\section{Controller}
\label{implemenation:controller}

The controller is a data sharing node that is responsible for providing a dataset for data processing. It listens for processing requests and forwards datasets to data processors. Is implemented as described in §~\ref{solution:entities:data_controller} and §~\ref{solution:flow:pr_req}.

\section{Processor}
\label{implemenation:processor}

The processor is a data sharing node that listens to data processing notifications pushed by a data controller. It processes datasets and produces Zero Knowledge proofs of correctness of computation. Is implemented as described in §~\ref{solution:entities:data_processor} and §~\ref{solution:flow:pr_data}.

\section{Libraries}
\label{implemenation:libs}

Various libraries have been build to facilitate the interaction with the various components of the system. Each one is analyzed in the following sections.

\subsection{Blockchain}
\label{implemenation:libs:bl}

The blockchain library offers a set of functionalities that makes the communication with any blockchain implementation easier. Is a wrapper library over blockchain specific libraries, such as Ethereum's \verb|web3|, and it's main purpose is to be blockchain agnostic allowing the use of different Blockchains. For the moment only the Ethereum blockchain is supported.

Abstract Functions:

\begin{itemize}
  \item \verb|isConnected|: Return true if a connection to a blockchain node exists and false otherwise.
  \item \verb|getProvider|: Get the current provider (blockchain node)
  \item \verb|setProvider|: Set provider (blockchain node)
  \item \verb|getBalance|: Get the balance of an address
  \item \verb|setDefaultAccount|: Set the default address
  \item \verb|getDefaultAccount|: Get the default address
  \item \verb|getAccounts|: Get all accounts
  \item \verb|getLibInstance|: Get the instance of the blockchain library
  \item \verb|getFilter|: Get a filter object
  \item \verb|toBytes|: Covert a string to bytes
  \item \verb|fromBytes|: Covert bytes to a string
  \item \verb|registerDataSet|: Register a dataset
  \item \verb|registerProcessor|: Register a processor
  \item \verb|registerController|: Register a controller
  \item \verb|requestProcessing|: Request a data processing
  \item \verb|notifyProcessor|: Notify a data processor for data processing
\end{itemize}

\subsection{Crypto}
\label{implemenation:libs:cr}

The crypto library provides various cryptographic functionalities. It wraps the \verb|SJCL| and the \verb|Node.js Crypto module| libraries.

Functions:

\begin{itemize}
  \item \verb|generateKeyPair|: Generate an asymmetric encryption key pair
  \item \verb|generateKey|: Generate a symmetric encryption key
  \item \verb|pubEncrypt|: Encrypt a string with a public key with the use of ECC
  \item \verb|pubDecrypt|: Decrypt a ciphertext with a secret key with the use of ECC
  \item \verb|encryptFile|: Encrypt a file with a symmetric key with the use of AES256
  \item \verb|decryptFile|: Decrypt a file with a symmetric key with the use of AES256
  \item \verb|bytesToHex|: Convert bytes to hex
  \item \verb|hextoBytes|: Convert hex to bytes
  \item \verb|hashFile|: Hash a file with the use of SHA256
  \item \verb|hash|: Hash an array of values with the use of SHA256
\end{itemize}

\section{Command-line interface}
\label{implemenation:cli}

A command-line interface that provides a set of useful commands for the data sharing application.

Usage:

\begin{verbatim}
  $ node data-cli <options>
\end{verbatim}

Options:

\begin{itemize}
  \item \verb|-v| or \verb|--version|: Output the version number
  \item \verb|-g| or \verb|--generate-keys|: Generate an asymmetric
  \item \verb|-k| or \verb|--generate-key|: Generate a symmetric key
  \item \verb|-d| or \verb|--dummy-file <file>|: Generate a 45MB dummy file
  \item \verb|-e| or \verb|--encrypt-file <file>|: Encrypt a file
  \item \verb|-a| or \verb|--evaluation <bytes>|: Evaluate gas cost of bytes on Ethereum
  \item \verb|-s| or \verb|--hash <hash>|: Hash an array of values with SHA256
  \item \verb|-f| or \verb|--hash-file <file>|: Hash a file with SHA256
  \item \verb|-h| or \verb|--help|: Output usage information
\end{itemize}

\section{Smart contracts}
\label{implemenation:contracts}

The data sharing application is governed by a smart contract deployed on the Ethereum network. The smart contract contains the logic of the application. Thought the smart contract one can register a new dataset, request for data processing, register a data controller or a data processor, and save a ZKP proof along with the output of the computation. The data sharing smart contract is publicly available and can be used by any other contract in the network.

\subsection{Data set registration}
\label{implemenation:contracts:data_reg}

A data controller registers a new dataset to the application by calling the \verb|registerDataSet| function with the following arguments:

\begin{itemize}
  \item \verb|hash|: The SHA256 of the raw data set. It serves as a unique identifier of the dataset: the \verb|dataSetID|.
  \item \verb|name|: The name of the dataset. It can be an arbitrary string
  \item \verb|location|: The URI of the location of the data set
  \item \verb|category|: The category of the data set. It can be an arbitrary string
  \item \verb|metaHash|: The SHA256 of the submitted values: \verb|hash|, \verb|name|, \verb|location| and \verb|category|
\end{itemize}

\begin{lstlisting}[language=Solidity, caption={Data set registration function}]
  function registerDataSet(
    bytes32 hash,
    bytes32 name,
    string location,
    bytes32 category,
    bytes32 metaHash
  ) public returns (bool success);
\end{lstlisting}

The address of the controller is taken from the \verb|msg.sender| field.

\subsection{Request for processing}
\label{implemenation:contracts:req_pr}

A request for data processing consists of two parts: Requesting for data processing and notifying the data processor of that request.

A data processing request is made by calling the \verb|requestProcessing| function with the following arguments:

\begin{itemize}
  \item \verb|_dataSetID|: The id of the dataset of interest
  \item \verb|algorithmID|: The id of the algorithm that the processor will apply over the data set
  \item \verb|pubKey|: The public key of the requestor
\end{itemize}

A unique identifier of the request is created by applying the \verb|SHA256| hash function to the concatenation of the dataset's id with the address of the requestor. In particular, as \verb|SHA256(_dataSetID, msg.sender)|. The address of the requestor is taken from the \verb|msg.sender| field.

In order to notify a data processor that a request for data processing is pending, a data controller calls the \verb|notifyProcessor| function with the following arguments:

\begin{itemize}
  \item \verb|_processorAddress|: The address of the data processor that will process the dataset
  \item \verb|_requestID|: The id of the request
  \item \verb|encryptedKey|: The encrypted symmetric key with which the dataset has been encrypted
\end{itemize}

\begin{lstlisting}[language=Solidity, caption={Request for processing functions}]
  function requestProcessing(
    bytes32 _dataSetID,
    bytes32 algorithmID,
    string pubKey
  ) public returns (bool success);

  function notifyProcessor(
    address _processorAddress,
    bytes32 _requestID,
    string encryptedKey
  ) public returns (bool success);

\end{lstlisting}

\subsection{Processor Registration}
\label{implemenation:contracts:reg_processor}

A data processor can be registered by a transaction that invokes the \verb|registerProcessor| function with the following inputs:

\begin{itemize}
  \item \verb|_processorAddress|: The address of the data processor
  \item \verb|name|: The name of the data processor. It can be an arbitrary string
  \item \verb|pubKey|: The public key of the data processor
\end{itemize}

Only the contract creator can execute the \verb|registerProcessor| function.

\begin{lstlisting}[language=Solidity, caption={Data processor registration function}]
  function registerProcessor(
    address _processorAddress,
    bytes32 name,
    string pubKey
  ) public returns (bool success);
\end{lstlisting}

\subsection{Data Controller Registration}
\label{implemenation:contracts:reg_processor}

Likewise, a data processor can be registered by a transaction that invokes the \verb|registerController| function with the following inputs:

\begin{itemize}
  \item \verb|_controllerAddress|: The address of the data controller
  \item \verb|name|: The name of the data controller. It can be an arbitrary string
  \item \verb|pubKey|: The public key of the data controller
\end{itemize}

Again, only the contract creator can execute the \verb|registerController| function.

\begin{lstlisting}[language=Solidity, caption={Data controller registration function}]
  function registerController(
    address _controllerAddress,
    bytes32 name,
    string pubKey
  ) public returns (bool success);
\end{lstlisting}

\subsection{General functions}
\label{implemenation:contracts:general}

Various general function are provided to facilitate the usage of the system.

\begin{itemize}
  \item \verb|getDataSetInfo|: Given a data set id it returns all the information of the specific data set
  \item \verb|getRequestInfo|: Given a request id it returns all the information of the specific request
  \item \verb|getController|: Given a data controller address it returns all the information of the specific data controller
  \item \verb|getProcessor|: Given a data processor address it returns all the information of the specific data processor
\end{itemize}

\begin{lstlisting}[language=Solidity, caption={General functions}]
  function getDataSetInfo(bytes32 _dataSetID)
    public
    view
    returns(
        bytes32 name,
        string location,
        bytes32 category,
        bytes32 metaHash,
        address controller
      );

  function getRequestInfo(bytes32 _requestID)
    public
    view
    returns(
      bytes32 dataSetID,
      address requestor,
      bool hasProof,
      bool processed,
      bytes32 algorithmID,
      string pubKey
  );

  function getController(address _controller)
    public
    view
    returns(
      bytes32 name,
      string pubKey
  );

  function getProcessor(address _processor)
  public
  view
  returns(
    bytes32 name,
    string pubKey
  );
\end{lstlisting}

\subsection{Zero Knowledge Proof}
\label{implemenation:contracts:zkp}

A data processor can save the ZKP proof and the output of the computation by calling the \verb|addProof| function with the following arguments:

\begin{itemize}
  \item \verb|_requestID|: The request id
  \item \verb|proof|: The proof in hex format
  \item \verb|output|: The encrypted output
\end{itemize}

\begin{lstlisting}[language=Solidity, caption={Data sharing application events}]

function addProof(
  bytes32 _requestID,
  string proof,
  string output
) public returns (bool success);

\end{lstlisting}
\subsection{Events}
\label{implemenation:contracts:events}

Every function that is invoked by a signed transaction emits an event notifying the occurrence of the action by a specific participant. Every node of the network can register and listen to this events.

In particular:

\begin{itemize}
  \item \verb|NewDataSet|: A data set is registered
  \item \verb|NewProvider|: A data controller is registered
  \item \verb|NewProcessor|: A data processor is registered
  \item \verb|NewRequest|: A request for data processing is made
  \item \verb|Process|: A notification for data processing is sent to a specific processor
\end{itemize}

\begin{lstlisting}[language=Solidity, caption={Data sharing application events}]

  event NewDataSet(
      bytes32 hash,
      bytes32 name,
      string location,
      bytes32 category,
      bytes32 metaHash,
      address controller
  );

  event NewController(
      address _controllerAddress,
      bytes32 name,
      string pubKey
  );

  event NewProcessor(
      address _processorAddress,
      bytes32 name,
      string pubKey
  );

  event NewRequest(
      bytes32 _requestID,
      bytes32 _dataSetID,
      address _requestor,
      bytes32 algorithmID,
      string pubKey
  );
  event Process(
    address _processorAddress,
    bytes32 _requestID,
    string encryptedKey
  );
\end{lstlisting}

\section{Zero Knowledge Proofs}
\label{implemenation:zkp}

To generate and verify zkSNARKs proofs the \verb|Pequin| library~\cite{pequin} is used. \verb|Pequin| is a toolchain to verifiably execute programs expressed in the C programming language. \verb|Pequin| consists of a front-end and a back-end. The front-end takes C programs and transforms them to a set of arithmetic constraints as described in ~\ref{zkp:snarks}. The back-end in Pequin is a zk-SNARK. Pequin uses \verb|SCIPR Lab's libsnark|~\cite{libsnark}, which is an optimized implementation of the back-end of \verb|Pinocchio|~\cite{pinocchio-nearly-practical-verifiable-computation}, itself a refinement and implementation of \verb|GGPR|~\cite{ggpr}.

Each of the supported algorithm (§~\ref{solution:algorithms}) is written in the C programming language. A trusted setup has been run and a key pair $(e_k, v_k)$ has been generated for each of the available algorithms. The evaluation keys and the compiled programs are available to all data processors as they are necessary components for proof generation and data computation. The verification keys are also publicly available to all requestors important to proof verification.

A set of various bash scripts are made to facilitate the data processors and the data requestors. They provide simple commands that abstract the interaction with the \verb|Pequin| library.

\section{Programming details}
\label{implemenation:details}

The data sharing ecosystem (§~\ref{solution}) required various frameworks, technologies, programming languages and libraries.

In particular:

\begin{enumerate}
  \item RESTful API: \verb|Node.js v8.9.4|~\cite{nodejs}, \verb|Express 4.16.2|~\cite{expressjs}
  \item Command-line interface: \verb|Node.js v8.9.4|
  \item Distributed application: \verb|React 15.6.1|~\cite{react}, \verb|Redux 3.7.2|~\cite{redux} and \verb|Bootstrap 4.0.0|~\cite{bootstrap}
  \item Processor node: \verb|Node.js v8.9.4|
  \item Controller node: \verb|Node.js v8.9.4|
  \item Libraries:
    \begin{enumerate}
      \item Blockchain: \verb|web3 0.20.4|~\cite{web3js}
      \item Crypto: \verb|SJCL 1.0.7|~\cite{sjcl} and \verb|Node.js v8.9.4 Crypto Module|
    \end{enumerate}
  \item Smart contracts: \verb|Truffle 4.1.0|~\cite{truffle} and \verb|Solidity 0.4.24|~\cite{solidity}
  \item Zero Knowledge Proofs: \verb|Pequin|~\cite{pequin} and \verb|libsnark|~\cite{libsnark}
  \item Ethereum Blockchain: \verb|Ganache CLI v6.0.3 (ganache-core: 2.0.2)|~\cite{ganache}
\end{enumerate}

All code written in \verb|JavaScript| follows the \verb|ES6 (ECMAScript 2015)|~\cite{ECMA_2015} standard and the  \verb|StandardJS|~\cite{stdjs} style guide.

\begin{figure}[ht!]
  \center
  \begin{tikzpicture}
    \begin{scope}[node distance=2cm]
      \node[entity] (api) {API};
      \node[entity] (dapp) [above=of api] {Dapp};
      \node[entity] (eth_node) [below=of api] {Ethereum node};
      \node[database] (db) [right=of api, yshift=-0.3cm] {Mongodb};
    \end{scope}

    \draw[com] (api) -- (eth_node);
    \draw[com] (api) -- (dapp);
    \draw[com] (api.east) ++ (right:0ex) -- ++ (right:2cm);

    \node[block,fit=(dapp) (eth_node) (db), label={Local Enviroment}] (local) {};

    \node[barstyle={local}{Ethereum Blockchain}, below=of local] (blockchain) {};

    \draw[com] (eth_node.south) ++ (south:0ex) -- ++ (south:1.35cm);

  \end{tikzpicture}
  \caption{REST API \& Dapp implemenation scheme}
  \label{fig:implemenation:rest}
\end{figure}

\begin{figure}[ht!]
  \center
  \begin{tikzpicture}

    \begin{scope}[node distance=2cm]
      \node[entity] (controller) {Controller};
      \node[entity] (eth_node) [below=of controller] {Ethereum node};
      \node[database] (db) [right=of controller, yshift=-0.3cm] {Mongodb};
    \end{scope}

    \draw[com] (controller) -- (eth_node);
    \draw[com] (controller.east) ++ (right:0ex) -- ++ (right:2cm);

    \node[block,fit=(controller) (eth_node) (db), label={Local Enviroment}] (local) {};

    \node[barstyle={local}{Ethereum Blockchain}, below=of local] (blockchain) {};

    \draw[com] (eth_node.south) ++ (south:0ex) -- ++ (south:1.35cm);

    \begin{scope}[node distance=2cm]
      \node[entity] (eth_node_pr) [below=of blockchain, xshift=-2cm, yshift=0.65cm] {Ethereum node};
      \node[entity] (processor) [below=of eth_node_pr] {Processor};
      \node[entity] (zkp) [below=of processor] {ZKP};
      \node[database] (db_pr) [right=of processor, yshift=-0.3cm] {Mongodb};
    \end{scope}

    \draw[com] (eth_node_pr.north) ++ (north:0ex) -- ++ (north:1.35cm);
    \draw[com] (processor) -- (eth_node_pr);
    \draw[com] (processor) -- (zkp);
    \draw[com] (processor.east) ++ (right:0ex) -- ++ (right:2cm);

    \node[block,fit=(eth_node_pr) (processor) (db_pr) (zkp), label={below:Local Enviroment}] (local) {};

  \end{tikzpicture}
  \caption{Data processor and data controller implemenation scheme}
  \label{fig:implemenation:con_pr}
\end{figure}
