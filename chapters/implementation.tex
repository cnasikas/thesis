%!TEX root = ../thesis.tex

\tikzstyle{database} = [cylinder, shape border rotate=90, aspect=0.5, draw, fill=green!30],
\tikzstyle{node} = [rectangle, minimum width=4cm, minimum height=1cm, text centered, draw=black, draw, fill=blue!30],
\tikzstyle{blockchain} = [draw,rectangle,minimum width=\textwidth, minimum height=1cm,anchor=west, fill=orange!30],
\tikzstyle{process} = [fill=red!30],
\tikzstyle{txt} = [],
\tikzstyle{entity} = [draw, minimum height=1cm, minimum width=4cm, fill=blue!30]
\tikzstyle{com} = [<->]
\tikzstyle{block} = [draw,thick,inner sep=10pt]
\tikzset{barstyle/.style 2 args={
      draw,
      minimum height=3em,
      fill=orange!30,
      fit={(#1.west) (#1.east)}, inner sep=0pt, label={center:#2}
    }
}

\chapter{Implementation}
\label{implemenation}

\section{RESTful API}

A RESTful API is provided to facilitate communication between any application, that follows the REST architecture, and the data sharing Blockchain ecosystem. The REST API expose the blockchain business network that can be easily consumed by HTTP or REST clients. That way, any developer familiar with existing web technologies and frameworks is not obligated to learn the interval mechanisms of a Blockchain system to develop and deploy applications atop.

The REST API expose a set of routes each one having an HTTP method and a URI.

\begin{table}[ht!]
\centering
\begin{tabular}{|l|l|}
\hline
 Method & URI  \\ \hline
 GET & /\  \\ \hline
 GET &  /contracts \\ \hline
 GET &  /datastore \\ \hline
 GET &  /datastore/\{data\} \\ \hline
 POST &  /datastore\\ \hline
 GET &  /accounts \\ \hline
 GET &  /accounts/\{account\} \\ \hline
 GET &  /requests \\ \hline
 GET &  /requests/\{request\} \\ \hline
 POST &  /requests \\ \hline
 GET &  /processors \\ \hline
 POST &  /processors \\ \hline
\end{tabular}
\caption{RESTful API Routes}
\label{table:api_routes}
\end{table}

\section{Dapp}
\section{Processor}
\section{Controller}
\section{Command-line interface}
\section{Smart contracts}

\begin{figure}[ht!]
  \center
  \begin{tikzpicture}
    \begin{scope}[node distance=2cm]
      \node[entity] (api) {API};
      \node[entity] (dapp) [above=of api] {Dapp};
      \node[entity] (eth_node) [below=of api] {Ethereum node};
      \node[database] (db) [right=of api, yshift=-0.3cm] {Mongodb};
    \end{scope}

    \draw[com] (api) -- (eth_node);
    \draw[com] (api) -- (dapp);
    \draw[com] (api.east) ++ (right:0ex) -- ++ (right:2cm);

    \node[block,fit=(dapp) (eth_node) (db), label={Local Enviroment}] (local) {};

    \node[barstyle={local}{Ethereum Blockchain}, below=of local] (blockchain) {};

    \draw[com] (eth_node.south) ++ (south:0ex) -- ++ (south:1.35cm);

    \begin{scope}[node distance=2cm]
      \node[entity] (eth_node_pr) [below=of blockchain, xshift=-2cm, yshift=0.65cm] {Ethereum node};
      \node[entity] (processor) [below=of eth_node_pr] {Processor};
      \node[database] (db_pr) [right=of processor, yshift=-0.3cm] {Mongodb};
    \end{scope}

    \draw[com] (eth_node_pr.north) ++ (north:0ex) -- ++ (north:1.35cm);
    \draw[com] (processor) -- (eth_node_pr);
    \draw[com] (processor.east) ++ (right:0ex) -- ++ (right:2cm);

    \node[block,fit=(eth_node_pr) (processor) (db_pr), label={below:Local Enviroment}] (local) {};

  \end{tikzpicture}
  \caption{Implementation}
  \label{fig:implemenation}
\end{figure}
