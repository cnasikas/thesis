%!TEX root = ../thesis.tex
\chapter{Future Work}
\label{future_work}

\section{Secure Multi party computation}
\label{future_work:mpc}

In the setting of multiparty computation~\cite{Ben-Or:1988:CTN:62212.62213}, sets of two or more parties with private inputs wish to jointly compute some (predetermined) function of their inputs~\cite{mpc}. Secure multiparty computation assume malicious behavior by a subset of participant entities.

The two important requirements of any secure computation protocol are privacy and correctness. The privacy requirement states that the parties should not learn anything else than the output of the computation and the correctness requirement states that that each party should receive its correct output~\cite{mpc}.

Informally, consider $n$ parties with private inputs $x_1, x_2, \dots$. The parties want to compute the outcome of the function $f(x_1, x_2, \dots)$ where the respective inputs remain private.

The first secure multiparty computation problem was described by Yao in~\cite{Yao:1982:PSC:1398511.1382751} and is called the Yao's Millionaires' problem. The problem discusses two millionaires, Alice and Bob, who are interested in knowing which of them is richer without revealing their actual wealth. Specifically, Yao’s millionaires’ problem is the problem of computing the predicate, $a \geq b$ where $a$ is Alice's holding and $b$ Bob's holding, without disclosing anything more than the result to either party~\cite{mpc_ioannidis}.

Multi-party computation protocols a have a wide range of applications. They can be used in voting systems where each party votes for a candidate and they want to compute the winner of the voting without revealing their vote. Another use case it that of auctions. Several parties are bidding for a product where the winning party and maximum bid should be determined, without revealing bids of other parties

Most MPC protocols make use of secret sharing. A secret sharing scheme allows a value to be shared among $n$ parties where some of the parts or all of them are needed in order to reconstruct the secret~\cite{Kamm:2015:ASM:2836836}. Usually, there is a threshold $t$ so that at least $t$ part of the secret are needed to reconstruct the original secret.

Using an MPC protocol based on secret sharing the need for transferring the symmetric key to the data processor and the decryption of the data (§~\ref{solution:flow:pr_req}, §~\ref{solution:flow:pr_data})---so the processing can be applied---can be eliminated.

Data queries can be computed in a distributed way, using an MPC cluster of data processors. Data can be split between data processors without having access to the data in its entirety~\cite{DBLP:journals/corr/ZyskindNP15}.

This way, the need for a trusted processor is eliminated as the data processors do not have access to the unciphered dataset and still process the dataset in a privacy-preserving manner.

\section{Fees}
\label{future_work:fees}

As discussed in~\ref{solution:treat_model:mrequestor} a malicious requestor can attack the system by overflowing the network with multiple requests aiming to prevent other request to be fulfilled. A scalable payment system could be used as a countermeasure to this type of attacks where a requestor pays a fixed price per byte processed. The price can decrease as the data size increases.

A payment system helps to prevent DDoS attacks as each request has cost and the attack becomes expensive. A rational malicious requestor has to significantly gain more than the cost of the attack to be profitable. Moreover, a payment system will incentivize the data processors to be honest and contribute to the prosperity of the network.

\section{Privacy Preserving Queries}
\label{future_work:ppq}

The algorithms (queries) the system supports are not privacy-preserving. The privacy of the individuals in the dataset must be protected with various privacy-preserving techniques such as differential privacy~\cite{differential_privacy}, ~\cite{Samarati98protectingprivacy} and l-diversity~\cite{Aggarwal2008}. This techniques can be either applied by the data controllers before handing over the dataset to the processor or by the processors themselves where a requestor wants the de-identification of the dataset as a processing procedure.

\section{Reputation system}
\label{future_work:ranking_system}

A decentralized reputation system~\cite{trust_is_risk}, resilient to Sybil attacks, could eliminate the need of a centralized trust model of a public key infrastructure (PKI), which relies on a trusted entity to authenticate, identify and verify the participants of the system. The benefits of such system are pseudonymity of participants, truly decentralization and trust is build by the system itself. The users in such systems built trust relationships among them and measure trustworthiness in a setting where assets can be exchanged between them~\cite{trust_is_risk}. A data controller, whom its dataset are of bad quality or fabricated with malicious intents, could be identified by the users of the system and gain bad reputation. It is evident, that the users of the network should choose datasets where their owners have high trust. The same applies to data processors and even the requestor themselves.

\section{Analytics}
\label{future_work:analytics}

\section{Consent}
\label{future_work:consent}

Modelling dynamic consent in smart contracts should be carefully analyzed taking into account design issues that are related to smart contracts lifecycle, required state variables for storing contract’s information, and access restrictions to those variables.

Neisse et al.~\cite{DBLP:journals/corr/NeisseSF17} proposed the following three models for data accountability and provenance tracking that comply with the GDPR(§~\ref{problem:regulations}):

\begin{enumerate}
  \item Contract for a specific controller: A contract where the data subject creates a contract tailored for each data controller
  \item Contract for specific data: A contract where each subject’s data instance is shared among all data controllers
  \item Contract for multiple data subjects: A contract where the data controller specifies how the data from all subjects are treated from the controller
\end{enumerate}

The first contract is more adequate for sensitive data~\cite{DBLP:journals/corr/NeisseSF17}, such as health data, since a subject-controller relationship~\cite{Azaria2016} is being created where each patient has a different contract for each controller accessing their data, providing fine grained access control and provenance information. On the other hand, it has the highest cardinality among the others two types~\cite{DBLP:journals/corr/NeisseSF17}. The second contract suffers from direct linkability as a unique subject address is needed compromising patients’ privacy~\cite{DBLP:journals/corr/NeisseSF17}. Lastly, the third one has the lowest cardinality among the others but also has the lower level of customization due to the limited number of contract options~\cite{DBLP:journals/corr/NeisseSF17}.
