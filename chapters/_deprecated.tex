\subsection{One-time pad}
\label{preliminaries:sym:otp}

A one-time pad (OTP) is an symmetric encryption scheme where the keys, messages, and ciphertexts are of the same length and the key is never reused or repeated. The scheme is defined as follows~\cite{kiagias:crypto, boneh_crypto}:

\begin{itemize}
  \item Space: $\calm = \calc = \calk = \{0, 1\}^{n}$
  \item Encryption: $Enc_k(m) = k \xor m$
  \item Decryption: $Dec_k(c) = k \xor c$
\end{itemize}

The one-time pad is a \textbf{perfectly secure} scheme~\cite{kiagias:crypto, boneh_crypto}. The main drawback lies in the fact that the key must be at least the length of the original message. If a party wants to encrypt and sent 1GB video to another party, they must have already share a 1GB key. It is proven~\cite{shannon_otp} that perfect security (or perfect secrecy) can only be achieved when the key size is at least as long as the size of the message. As a consequence, perfect security schemes are impossible to use in practice.

\subsection{Stream Ciphers}
\label{preliminaries:sym:stream}

In~\ref{preliminaries:sym:otp} we saw that a symmetric scheme to be perfectly secure, keys and messages must have at least the same length. However, with weaker notion of security we can create symmetric encryption schemes where the size of the key $k$ is much shorter than the original message. Stream ciphers can encrypt or decrypt arbitrary long messages with small fixed length size keys. Instead of using a key of size $n$, a shorter seed $s$ of $l$-bits, where $l$ is much smaller than $n$, is used as the symmetric key. Then from the seed $s$ a $n$-bit key is derived and is used for encryption and decryption. The seed $s$ is expanded to $n$ by a deterministic polynomial algorithm $G$ that maps $l$-bit strings to $n$-bit strings. The algorithm $G$ is called \textbf{pseudorandom generator} (PRG) and the following condition must hold for any PRG~\cite{Katz:2014:IMC:2700550}:

\begin{itemize}
  \item Expansion: On input $s \in \{0, 1\}^{l}$ outputs $k \in \{0, 1\}^{n}$ where $\forall l: n > l$
  \item Pseudorandomness: Lets $\cald$ be a distribution over strings of length $n$. $\cald$ is indistinguishable from the uniform distribution over strings of length $n$. In other words, it is infeasible for any polynomial-time algorithm to tell whether it is given a string sampled according to $\cald$ or uniformly at random.
\end{itemize}

So, with the use of PRGs we can construct encryption schemes with short key sizes and long messages. A stream chipher is defined as follows:

\begin{itemize}
  \item The key generation algorithm $\calg$: Takes as input a security parameter $1^{n}$ and outputs a seed $s \in \{0, 1\}^{l}$ chosen uniformly at random.
  \item The encryption algorithm $\cale$: Takes as input a seed $s \in \{0, 1\}^{l}$ and a message $m \in \{0, 1\}^{n}$ and outputs the ciphertext $c = Enc_s(m) = G(s) \xor m$
  \item The decryption algorithm $\cald$: Takes as input a seed $s \in \{0, 1\}^{l}$ and a cipher $c \in \{0, 1\}^{n}$ and outputs the plaintext $m = Dec_s(c) = G(s) \xor c$
\end{itemize}

An important point, is that $s$ must be chosen uniformly at random and be long enough to withstand brute-force attacks.

There is a number of practical constructed stream ciphers. The most wide used are Salsa20~\cite{Salsa20}, ChaCha~\cite{chacha} and RC4.

\subsection{Enigma}\label{blockchain:impl:enigma}

Enigma is a decentralized computation platform with guaranteed privacy~\cite{DBLP:journals/corr/ZyskindNP15}. It operates through a peer-to-peer network enabling
different parties to store and run computation on data while keeping privacy. Enigma uses a highly optimized version of secure multi-party computation (MPC)
guaranteed by a verifiable secret-sharing scheme~\cite{DBLP:journals/corr/ZyskindNP15}. A blockchain is utilized as the controller of the network, managing access to the data and identities.
The data are stored off-chain, encrypted in a distributed database, and a modified distributed hash table (DTH), accessible through the blockchain, is used for holding only references to the data.
