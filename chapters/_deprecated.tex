\subsection{One-time pad}
\label{preliminaries:sym:otp}

A one-time pad (OTP) is an symmetric encryption scheme where the keys, messages, and ciphertexts are of the same length and the key is never reused or repeated. The scheme is defined as follows~\cite{kiagias:crypto, boneh_crypto}:

\begin{itemize}
  \item Space: $\calm = \calc = \calk = \{0, 1\}^{n}$
  \item Encryption: $Enc_k(m) = k \xor m$
  \item Decryption: $Dec_k(c) = k \xor c$
\end{itemize}

The one-time pad is a \textbf{perfectly secure} scheme~\cite{kiagias:crypto, boneh_crypto}. The main drawback lies in the fact that the key must be at least the length of the original message. If a party wants to encrypt and sent 1GB video to another party, they must have already share a 1GB key. It is proven~\cite{shannon_otp} that perfect security (or perfect secrecy) can only be achieved when the key size is at least as long as the size of the message. As a consequence, perfect security schemes are impossible to use in practice.

\subsection{Stream Ciphers}
\label{preliminaries:sym:stream}

In~\ref{preliminaries:sym:otp} we saw that a symmetric scheme to be perfectly secure, keys and messages must have at least the same length. However, with weaker notion of security we can create symmetric encryption schemes where the size of the key $k$ is much shorter than the original message. Stream ciphers can encrypt or decrypt arbitrary long messages with small fixed length size keys. Instead of using a key of size $n$, a shorter seed $s$ of $l$-bits, where $l$ is much smaller than $n$, is used as the symmetric key. Then from the seed $s$ a $n$-bit key is derived and is used for encryption and decryption. The seed $s$ is expanded to $n$ by a deterministic polynomial algorithm $G$ that maps $l$-bit strings to $n$-bit strings. The algorithm $G$ is called \textbf{pseudorandom generator} (PRG) and the following condition must hold for any PRG~\cite{Katz:2014:IMC:2700550}:

\begin{itemize}
  \item Expansion: On input $s \in \{0, 1\}^{l}$ outputs $k \in \{0, 1\}^{n}$ where $\forall l: n > l$
  \item Pseudorandomness: Lets $\cald$ be a distribution over strings of length $n$. $\cald$ is indistinguishable from the uniform distribution over strings of length $n$. In other words, it is infeasible for any polynomial-time algorithm to tell whether it is given a string sampled according to $\cald$ or uniformly at random.
\end{itemize}

So, with the use of PRGs we can construct encryption schemes with short key sizes and long messages. A stream chipher is defined as follows:

\begin{itemize}
  \item The key generation algorithm $\calg$: Takes as input a security parameter $1^{n}$ and outputs a seed $s \in \{0, 1\}^{l}$ chosen uniformly at random.
  \item The encryption algorithm $\cale$: Takes as input a seed $s \in \{0, 1\}^{l}$ and a message $m \in \{0, 1\}^{n}$ and outputs the ciphertext $c = Enc_s(m) = G(s) \xor m$
  \item The decryption algorithm $\cald$: Takes as input a seed $s \in \{0, 1\}^{l}$ and a cipher $c \in \{0, 1\}^{n}$ and outputs the plaintext $m = Dec_s(c) = G(s) \xor c$
\end{itemize}

An important point, is that $s$ must be chosen uniformly at random and be long enough to withstand brute-force attacks.

There is a number of practical constructed stream ciphers. The most wide used are Salsa20~\cite{Salsa20}, ChaCha~\cite{chacha} and RC4.

\subsection{Diffie–Hellman key exchange}
\label{preliminaries:pub:dh}

The protocol works as follows~\cite{Katz:2014:IMC:2700550, kiagias:crypto}:

\begin{enumerate}
  \item Alice and Bob  with the use of a group generation algorithm $\calg$ agree on the description of a finite group $\G$ with input $1^{n}$. The common input for Alice and Bob is the tuple $(p, m, g)$ where $p$ is a large prime and $g$ is the generator of the finite group $\G$ of order $m$.
  \item Alice choose a random index $x_a \rselect{\Z_m}$ and computes $y_a \leftarrow{g^{x_a}}modp$. Alice sends $y_a$ to Bob.
  \item Bob choose a random index $x_b \rselect{\Z_m}$ and computes $y_b \leftarrow{g^{x_b}}modp$. Alice sends $y_b$ to Bob.
  \item Alice outputs $k = y_b^{x_a}modp = g^{{x_a}{x_b}}modp$
  \item Bob outputs $k = y_a^{x_b}modp = g^{{x_a}{x_b}}modp$
\end{enumerate}

The security of Diffie–Hellman key exchange is based on the difficulty of the discrete log problem (DLOG), which is the problem of computing $x$ given $g^{x}$ in a cyclic group $\G$. A passive adversary cannot compute the private key $k$ because she does not know $x_a$ or $x_b$. To find them she have to compute the discrete log which is assumed to be hard.

\begin{figure}[!hb]
  \centering
  \begin{tikzpicture}
    \matrix (m)[matrix of nodes, column  sep=2cm,row  sep=4mm, nodes={draw=none, anchor=center,text depth=0pt} ]{
    Alice & & Bob\\
    $x_a \rselect{\Z_m}$ & & $x_b \rselect{\Z_m}$ \\
    $y_a \leftarrow{g^{x_a}}modp$ & & $y_b \leftarrow{g^{x_b}}modp$ \\
     & $y_a$ & \\
     & $y_b$ & \\
     $k = y_b^{x_a}modp$ & & $k = y_a^{x_b}modp$ \\
    };

    \draw[shorten <=-1.5cm,shorten >=-1.5cm] (m-1-1.south east)--(m-1-1.south west);
    \draw[shorten <=-1.5cm,shorten >=-1.5cm] (m-1-3.south east)--(m-1-3.south west);
    \draw[shorten <=-1cm,shorten >=-1cm,-latex] (m-4-2.south west)--(m-4-2.south east);
    \draw[shorten <=-1cm,shorten >=-1cm,-latex] (m-5-2.south east)--(m-5-2.south west);

  \end{tikzpicture}
  \caption{Diffie–Hellman key exchange}
  \label{fig:crypto:dh}
\end{figure}

\subsection{The RSA Cryptosystem}
\label{preliminaries:pub:rsa}

The RSA cryptosystem~\cite{rsa} was developed in 1977 at MIT by Ron Rivest, Adi Shamer, and Leonard Adleman and is still the most widely used. It was the first public-key encryption scheme that could both encrypt and sign messages~\cite{kiagias:crypto}.

It works as follows~\cite{Katz:2014:IMC:2700550, kiagias:crypto}:

\begin{itemize}
  \item Key Generation:
    \begin{enumerate}
      \item Select randomly to large primes $p, q$ of length $n$ bits
      \item Compute $N = p*q$
      \item Calculate $\phi(N) = (p - 1)(q - 1)$
      \item Find $e$ such that $gcd(e, \phi(N)) = 1$
      \item Compute $d = e^{-1} mod\phi(N)$
      \item Public key is $(N, e)$ and private key is $(N, d)$
    \end{enumerate}
  \item Encryption: On input a public key $p_k = (N, e)$ and a message $m$ it computes the ciphertext $c$ as $ Enc_{p_k}(m) = m^{e}modN$
  \item Decryption: On input a private key $s_k = (N, d)$ and a ciphertext $c$ it computes the message $m$ as $ Dec_{s_k}(c) = c^{d}modN = m^{ed}modN = m$
\end{itemize}

If $p, q$ are known or obvious any interested party can compute $\phi(N)$ and therefore $d$. The RSA Cryptosystem is secure under the assumption that factorization of $N$ is believed to be hard.

The above mention protocol is not secure as the encryption function is deterministic and for the same input is produces the same output. As mention, a deterministic cryptosystem is not secure. One way to randomize the encryption function, is by appending random padding to message.

\subsection{The El Gamal Cryptosystem}
\label{preliminaries:pub:el_gamal}

The El Gamal Cryptosystem~\cite{el_gamal} is another popular and wide-used encryption scheme. The security of the system is based on the hardness of discrete log problem. It works as follows~\cite{Katz:2014:IMC:2700550, kiagias:crypto}:

\begin{itemize}
  \item Key generation:
    \begin{enumerate}
        \item Run a group generation algorithm $\calg$ to produce the description of a finite group $\G$ with input $1^{n}$. The output is the tuple $(p, m, g)$ where $p$ is a large prime and $g$ is the generator of the finite group $\G$ of order $m$.
        \item Select randomly $x \rselect{\Z_m}$
        \item Calculate $h = g^{x}modp$
        \item The public key is $((p, m, g), h)$
        \item The secret key is $x$
    \end{enumerate}
  \item Encryption: Encrypts a message $m \in \G$
    \begin{enumerate}
      \item Choose randomly $r \rselect{\Z_m}$
      \item Compute $G = g^{r}modp$
      \item Compute $M = mh^{r}modp$
      \item Return $c = (G, M)$
    \end{enumerate}
  \item Decryption: Decrypts a ciphertext $c = (G, M)$
    \begin{enumerate}
      \item Compute $m = M / G^{x} modp$
      \item Return $m$
    \end{enumerate}
\end{itemize}

Another way to express the security of El Gamal scheme is under the decisional Diffie Hellman problem (DDH) which states that the tuples $(g^a, g^b, g^c)$ and $(g^a, g^b, g^{ab})$ are indistinguishable by a probabilistic polynomial-time (PPT) adversary.

\subsection{RSA signatures}
\label{preliminaries:sign:rsa}

As describes in~\ref{preliminaries:pub:rsa} the RSA cryptosystem can be used to sign messages. The RSA digital signature scheme works as follows~\cite{Katz:2014:IMC:2700550, kiagias:crypto}:

\begin{itemize}
  \item Key Generation:
    \begin{enumerate}
      \item Select randomly to large primes $p, q$ of length $n$ bits
      \item Compute $N = p*q$
      \item Calculate $\phi(N) = (p - 1)(q - 1)$
      \item Find $e$ such that $gcd(e, \phi(N)) = 1$
      \item Compute $d = e^{-1} mod\phi(N)$
      \item Verification key is $(N, e)$ and signing key is $(N, d)$
    \end{enumerate}
  \item Sign: On input a sign key $s_k = (N, d)$ and a message $m \in \Z^{*}_{N}$ it computes the signature $\sigma$ as $ Sign_{s_k}(m) = m^{d}modN$
  \item Verify: On input a verification key $p_k = (N, e)$, a message $m \in \Z^{*}_{N}$ and a signature $\sigma \in \Z^{*}_{N}$ it outputs $1$ if and only if $m \stackrel{?}{=} Verify_{p_k}(\sigma, m) = \sigma^{e}modN$
\end{itemize}

The above signature scheme is not secure as an adversary can forge a sign, based on the public key alone, by choosing an arbitrary $\sigma \in \Z^{*}_{N}$ and compute $m = \sigma^{e}modN$. Another attack on the RSA signature scheme allows the adversary to output a forgery on any message of the adversary's choice. One proposal to protect against those attacks, that can be proven secure under certain assumptions, is by applying a cryptographic hash function $H$ to the message before sign it. The minimal requirement for the system to be secure is that $H$ must be collision-resistant.

\subsection{Diffie-Hellman key exchange}
\label{preliminaries:el_curves:dh}

The elliptic curve analogue of Diffie-Hellman key exchange protocol (ECDH) defined in~\ref{preliminaries:pub:dh} can implemented as follows~\cite{elliptic_curves}:

\begin{enumerate}
  \item Alice and Bob agree on elliptic curve $E$ over a finite field $F_p$ for a prime $p$ such that the discrete log problem is hard in $E$ over $F_p$. They also agree on a point $P$ on $E$.
  \item Alice choose a secret integer $a$ and computes $P_a = aP$. Alice sends $P_a$ to Bob
  \item Bob choose a secret integer $b$ and computes $P_b = bP$. Bob sends $P_b$ to Alice
  \item Alice computes $k = aP_b = abP$
  \item Bob computes $k = bP_a = abP$
\end{enumerate}

The security of the protocol is based on ECDLP (§~\ref{preliminaries:el_curves:key_gen}).

\subsection{The El Gamal Cryptosystem}
\label{preliminaries:el_curves:el_gamal}

The elliptic curve analogue of El Gamal cryptosystem defined in~\ref{preliminaries:pub:el_gamal} can implemented as follows~\cite{elliptic_curves_2}:

\begin{itemize}
  \item Key generation: Run elliptic curve key generation algorithm defined in~\ref{preliminaries:el_curves:key_gen} and get a key pair $(Q, d)$.
  \item Encryption: Encrypts a message $m$ represented as a point $M$ in $E$ over $\F_p$
    \begin{enumerate}
      \item Choose randomly $r \rselect{\in \Z_q}$
      \item Compute $G = rP$
      \item Compute $C = M + rQ$
      \item Return $c = (G, C)$
    \end{enumerate}
  \item Decryption: Decrypts a ciphertext $c$
    \begin{enumerate}
      \item Compute $M = C - dG$, and extract $m$ from $M$
      \item return $M$
    \end{enumerate}
\end{itemize}

An eavesdropper adversary who wants to obtain $M$ needs to compute $rQ$; given $Q$, $P$ and $G = rP$ find $r \in Z_q$. This is the elliptic curve analogue of the Diffie-Hellman problem.

\subsection{Enigma}\label{blockchain:impl:enigma}

Enigma is a decentralized computation platform with guaranteed privacy~\cite{DBLP:journals/corr/ZyskindNP15}. It operates through a peer-to-peer network enabling
different parties to store and run computation on data while keeping privacy. Enigma uses a highly optimized version of secure multi-party computation (MPC)
guaranteed by a verifiable secret-sharing scheme~\cite{DBLP:journals/corr/ZyskindNP15}. A blockchain is utilized as the controller of the network, managing access to the data and identities.
The data are stored off-chain, encrypted in a distributed database, and a modified distributed hash table (DTH), accessible through the blockchain, is used for holding only references to the data.

\subsection{zksnarks}\label{zksnarks}

The main idea for constructing zkSNARKs is firstly to transform the generic program as a quadratic equation of polynomials (QAP)~\cite{ggpr}: $p(x)q(x) = s(x)r(x)$, where the equality holds if and only if the program is computed correctly. So the prover wants to convince the verifier that the equality holds.

Succinctness is achieved by random sampling. The verifier chooses a secret evaluation point $x_0$ to reduce the problem from polynomial multiplication and equality to simple number arithmetic checks: $p(x_0)q(x_0) = s(x_0)r(x_0)$. To allow the prover to compute the evaluation of the polynomials at $x_0$, without revealing $x_0$, Homomorphic encryption is used:  $E(p(x_0))E(q(x_0)) = E(s(x_0))E(r(x_0))$.

Finally the prover obfuscates the encrypted values by multiplying with a number so that the verifier can still check the correctness of the structure without knowing the actual encrypted values: $E(k + p(x_0))E(k + q(x_0)) = E(k + s(x_0))E(k + r(x_0))$.

% Introduction

In the big data era, huge amounts of data are constantly being collected and analyzed, evidently, leading innovation and
economic growth~\cite{10.1109/SPW.2015.27}. Data origins are various including wearable and IoT devices.
While data grows exponentially, gathering and storing them imposes high costs and liabilities.

Centralized organizations, amass large quantities of personal and sensitive information on which individuals have little or no control ~\cite{10.1109/SPW.2015.27}. Furthermore, there is a growing public concern about user privacy.

However, data can be private and protected by acts, such as GDPR, making data processing impossible.

Blockchain’s distributed nature eliminates the need for a trusted central authority and enables the connection between trustless entities.
The immutability and auditability of blockchain blocks and transactions, and cryptographical guarantees provided, can enforce liability, transparency and accountability without compromising privacy and security.
In addition, blockchain can significantly reduce data transaction costs between entities and increase transaction efficiency.
Like all technology, blockchain has limitations. It is argued that blockchain is not suited for high performance transactions or as a database replacement. Due to the decentralized nature of blockchain and its necessity for a consensus mechanism the transaction rate remains quite low compared to financial services~\cite{Sompolinsky2015,Zohar:2015:BUH:2817191.2701411}.
Blockchain is not made for big data. The amount of data that blockchain can store and process is very limited so off-chain data frameworks are needed.
