%!TEX root = ../thesis.tex
\chapter{Conclusion}
\label{conclusion}

As data are constantly gathered from various locations, the need of data processing have been emerged. However, various of these data are private and cannot be shared or exchanged. It is evident that data processing in a privacy preserving manner is challenging.

To address this issue we propose a platform where data controllers can register their datasets making them available for processing by data processors on behalf of data requestors. We use a distributed ledger as the controller of the system. The use of distributed ledgers could help to unite trustless entities with shared interests; the data processing of sensitive datasets. Utilizing the blockchain as the coordinator of the system where all actions of participants are recorded could providing trasparency, accountability, non-repudation, data provenance, and auditability as transactions are immutable. In addition to the blockchain, we use a Zero Knowledge verifiable computation scheme with which the data processors can produce a proof of correctness of computation upon a specific dataset and algorithm.

The combination of the blockchain with zero knowledge proofs enables the data processing in a privacy preserving manner. However, various obstacles remain to be bypassed. We aim to a fully decentralized data sharing ecosystem where researchers and organization can extract all the valuable informations of a dataset without compromising the individuals privacy.
