%!TEX root = ../thesis.tex
\SetBlockThreshold{1}
\chapter{Problem Statement}
\label{problem}

\section{Overview}
\label{problem:overview}

The continuous discussions about Big Data over the past few years has drawn the attention of researchers and individuals who try to understand their complex meaning and effects in people's lives. The new Data Age we live in has surely caused great shifts in preexisting understandings of the world.  It has transformed the way information is produced, transmitted, processed and stored, thus altering its value as well as the whole mode of operation of the global markets.

The evolution of computer power processing, hardware capacity and, later,  cloud storage contributed to the creation of information economy. Technological advancements led to the creation of handheld devices able to run multiple applications, use sensors, and produce, transmit and store huge amounts of data. The amount of existing data is rapidly increasing and it is estimated that 20\% of the world's data have been collected over the past few years~\cite{10.1109/SPW.2015.27,big_data_better_worse}. Commercial use of data is only one of the many and diverse industries that big data have decisively influenced. Some examples of data intensive fields are marketing and advertising, healthcare, transportation, energy regulation and distribution, retail and demographics, and sensors  embedded in products for control, or IoT~\cite{big_data}.

The value of raw data varies from a hundred cents to over several hundred dollars per individual~\cite{pr_data_cost_1, pr_data_cost_2, pr_data_cost_3}. The more it is being analyzed, the more its value increases. As some put it~\cite{data_new_oil_01,data_new_oil_02,data_new_oil_03,data_new_oil_04,data_new_oil_05,data_new_oil_05,data_new_oil_06,data_new_oil_07,data_new_oil_08,data_new_oil_09}: data is the new oil. And like with all sources of wealth, they are related to multiple interests and, thus they should be given the attention required.

\eparn{Personal Data}
\label{problem:sym:overview:personal}

People constantly produce and publish data about themselves
leaving a constant moving digital fingerprint all over the Internet; their personal data. The type, quantity and value of personal data being collected are vast~\cite{emergence_new_assets_wef}: bank accounts, medical records, employment data, web searches, sites visited,
likes and dislikes, product purchase histories, quotes, tweets, texts, emails, phone calls, photos, videos, personal moments, emotions as well as coordinates of real-world locations.

Personal data can be gathered in three ways~\cite{emergence_new_assets_wef}:

\begin{itemize}
  \item Volunteered data: data shared explicitly by the individuals
  \item Observed data: data captured by recording actions of individuals
  \item Inferred data: data based on analysis of individuals
\end{itemize}

Big data cannot exist without personal input and user cooperation. Companies create free services in return for user input leading to consumer-driven big data collection.

Personal data are typically gathered by a few big tech companies that own and privatize them. This happens as personal data belong to the institutions that collects them.  This condition gave rise to the data broker and mining industry creating data marketplaces~\cite{dawex, q_dx, datastreamx} and trade platforms where data can be sold and bought for a price. All data is potentially for sale without the consent of the rightful owners, the users.

The existing centralized models in which third-parties collect and control vast amounts of personal data can be questionable. Individuals have little or no control over their personal data and what these data are used for. Public concern about user privacy and protection is raised as related new research is being published~\cite{10.1109/SPW.2015.27}. When personal data are not controlled by their producers, the rightful owners, individuals are not protected against market and state policies. The invasion of privacy is a severe offense that derives from unregulated data exploitation. All users connected to platforms and data harvesting applications should be empowered with the ability to own their personal data, to control how data are being collected, used, shared and by whom.

\eparn{Open Data}
\label{problem:sym:overview:open}

Nevertheless, the ability to have access and process large volumes of information can still be beneficial for individuals and the societies at large. Creating Open Data is an important goal of the research community and has a lot of advantages. By allowing researchers to perform their research on datasets referring to public data records accountability, fault detection and trust in the validity of prior research can be achieved~\cite{open_data_1}. The Open Data Movement~\cite{wiki:open_data} is a paradigm of free data distribution without limits, copyrights and patents. However, there is a growing public concern about privacy preservation of research participants, which can become subversive for data collection. As legislation in the form of privacy disclosure acts is forced, data become private, sensitive and protected. At the same time though, data processing is rendered impossible, inefficient or false.

\section{Regulations}\label{problem:regulations}

The EU Data Protection Directive 95/46/EC (DPR)~\cite{eu-46ec-1995} defines personal data as follows:
\blockquote{
personal data shall mean any information relating to an identified or identifiable natural person ('data subject'); an identifiable person is one who can be identified, directly or indirectly, in particular by reference to an identification number or to one or more factors specific to his physical, physiological, mental, economic, cultural or social identity;
}
However there has been a clear notion that the data subject can potentially be identified by pseudo-identifiers~\cite{pii}.
In the new General Data Protection Regulation (GDPR)~\cite{gdpr} forced by the EU this has been formalized as:
\blockquote{
a data subject is one who can be identified, directly or indirectly, by means reasonably likely to be used by the controller or by any other natural or legal person
}

The recent approval of GDPR~\cite{gdpr} in 2016 by the European Commission (EC)
imposes new obligations on data controllers and processors in contrast to the previously adopted Data Protection Directive (DPD)~\cite{eu-46ec-1995}.
New legislation aims in the extension of responsibility and accountability requirements of organizations also demanding explicit
consent of the data subject (person) while securing her right to withdraw, and to be forgotten~\cite{DBLP:journals/corr/NeisseSF17}.

Key changes of the new data protection law introduced in GDPR in contrast to the DPR are presented below~\cite{DBLP:journals/corr/NeisseSF17}:
\begin{itemize}
    \item organizations based outside EU that process personal data of EU residents are bound by the new legislation
    \item All EU member states are obliged to comply to a single set of rules
    \item The responsibility and accountability requirements of organizations are extending
    \item EU residents are empowered with explicit consent over their data, maintaining their right to be forgotten
\end{itemize}

Data controllers and processors are now required to demonstrate the implementation of technical and organizational measures they have taken in order to ensure data security~\cite{mhmd}. The GDPR attempts to increase self-responsibility, and therefore accountability, with a particular reference to the processing of sensitive data, such as health and medical data~\cite{mhmd}. Data controllers are required to take measures to ensure that any product or service they provide is fully in line with the key principles of GDPR.

Finally, the data subject is empowered with total control over her data. Consent rules and a number of fundamental rights such as the right to be forgotten have been reinforced or introduced in the GDPR. The data subject's will has become a priority.

Data controllers can outsource data processing activities to data processors. This relationship can be established though contracts or any binding legal act. Personal data should not be made available to third parties without the consent of the data subject. Data processors are not considered as a third-party but rather as an extension of the data controller. In this sense, any activity put in place by the processors with the approval and instructions of the controller is considered to be carried out within the premises of the organization itself~\cite{mhmd}. For example, a hospital may act as a data controller of collected personal data of the hospital's patients and outsource an anonymized operation to a trusted data processor without any need to ask anew for the patients' consent.

\section{Motivations for applications}\label{problem:motivations}

Blockchain is decentralized, which means that there is no central authority that regulates and governs the recording of the transactions in the ledger. The ledger is maintained by the network participants, in which participation is free, and the global state of the ledger is agreed by a consensus mechanism (§~\ref{blockchain:consensus_mechanisms}). Data on the blockchain are stored in every node of the network. so that each node keeps a copy of the public ledger. Due to data replication there is no single point of failure (SPOF) and the system becomes fault-tolerant and resistant to malicious and denial-of-service attacks (DoS attacks). The users of the network are empowered with total ownership over their assets -- ranging from a token, a contract, datasets, medical records, chain-of-evidence documents, or citizenship documents -- through a system that guarantees security even in the presence of malicious users.

Blockchain is immutable, meaning that all transactions are irreversible and they can not be altered or deleted once they are confirmed and recorded on the blockchain. The realized transactions are stored in chronological order specifying if a transaction A precedes a transaction B. That allows the creation of a distributed digital timestamping service. Timestamping is useful for example in financial contracts, establishing precedence for copyright and auction bids~\cite{bl_auditability}.

The combination of a timestamping service and the immutability property of the blockchain, along with its consensus mechanism, can lead to the creation of various important properties such as accountability~\cite{10.1007/978-3-540-46588-1_20}, auditability and non-repudiation~\cite{non_repudiation} -- the ability
to definitively verify authenticity of statements recorded in the blockchain~\cite{bl_auditability}. These properties make the blockchain a distributed immutable, tamper-proof, transparent and audit log that any attempt to tamper with it can be immediately evident and easily detectable. And the users are allowed to confirm and prove whether the service operates in the intended way or not.

Blockchain technology can be used to support a wide variety of applications including document provenance tracking, digital assets, financial services, copyright, voting, distributed storage, donation, education, medical records, data exchange, and Internet of Things (IoT). Distributed consent for research trials can increase anonymous data samples~\cite{ibm} allowing Big Data Analytics in compliance with the GDPR. In addition, blockchain can significantly reduce data transaction costs between entities and increase transaction efficiency. Blockchain technology can be seen as a data sharing and processing system where privacy is mandatory.

Like all technology, blockchain has limitations. It is argued that blockchain is not suited for high performance transactions or as a database replacement. Due to the decentralized nature of the blockchain and its necessity for a consensus mechanism the transaction rate remains quite low compared to financial services~\cite{Sompolinsky2015, Zohar:2015:BUH:2817191.2701411}. Blockchain is not made for big data. The amount of data that a blockchain can store and process is very limited so off-chain data frameworks are needed.

Private and sensitive data cannot be transmitted outside the premises of the organization responsible for them. Computations over the data must be made by the organization itself. Yet, organizations must be trusted to assure computation is done properly. Outputs of computations can be false either by a malfunction or intentionally. Zero knowledge verifiable computation schemes (§~\ref{zkp:vc}) can generate proofs of correct computation without revealing private inputs. The proofs can be verified efficiently and with almost no extra overhead~\cite{pinocchio-nearly-practical-verifiable-computation}. The ability to construct publicly verifiable proofs without exposing sensitive information is crucial to a data processing system which guarantees the privacy of the datasets.

In this thesis we try to address the privacy concerns stemming from data exploitation and data regulation described above. We attempt to utilize the blockchain and a Zero Knowledge verifiable computation scheme and adapt it to our needs in order to provide a solution which can enable data sharing and the data processing in a privacy preserving manner, making anyone accountable for their actions and ensuring proper computation.
