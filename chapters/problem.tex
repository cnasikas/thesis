%!TEX root = ../thesis.tex
\SetBlockThreshold{1}
\chapter{Problem Statement}
\label{problem}

The amount of data is rapidly increasing and it is estimated that 20\% of the world's data have been collected the past
few years~\cite{10.1109/SPW.2015.27,big_data_better_worse}. Data storage rapid growth contributes to a substantial shift in economy power and source of economic value. We have entered to the area of Big Data. The impact of Big Data is huge and the consequences are far reaching.

Data is a strategic asset that allows companies to be competitive. The value of raw data varies from a hundred cents to over several hundred dollars per individual~\cite{pr_data_cost_1, pr_data_cost_2, pr_data_cost_3}. The more is analyzed, modified and enriched, the more its value increasing.

People constantly producing and publishing data about themselves
leaving a constant moving digital fingerprint over the Internet; their personal data. The type, quantity and value of personal data
being collected are vast~\cite{emergence_new_assets_wef}: banks accounts, medical records, employment data, web searches, sites visited,
likes, dislikes, product purchase histories, quotes, tweets, texts, emails, phone calls, photos, videos, personal moments, emotions
as well as coordinations of real-world locations.

Personal data are gathered by few big tech companies being the rightful owners---personal data is owned by the institution that collects them---and privatizing them. This gave rise to the data broker and mining industry creating data marketplaces~\cite{dawex, q_dx, datastreamx} and trade platforms where data can be sold and bought for a price. All data is potentially for sale.

Personal data can be gathered as follows~\cite{emergence_new_assets_wef}:

\begin{itemize}
  \item Volunteered data: data shared explicitly by the individuals
  \item Observed data: data captured by recording actions of individuals
  \item Inferred data: data based on analysis of individuals volunteered or observed data
\end{itemize}

Big data can not exist without personal input and wide-scale cooperation. Free services by companies---in return for user input--- is an example of consumer-driven big data; leisure time transforms into productive work.

As some put it~\cite{data_new_oil_01,data_new_oil_02,data_new_oil_03,data_new_oil_04,data_new_oil_05,data_new_oil_05,data_new_oil_06,data_new_oil_07,data_new_oil_08,data_new_oil_09}: Data is the new oil.

Treating personal data as a natural resource is a huge problem and should be prevented.

The current centralized model in which third-parties collect and control massive amount of personal data should be questioned.
Individuals have little or no control over their personal data and how they are used growing that way a public concern about
user privacy.

Personal data should be in total control by the person that produce them, the rightful owner. Individuals should be empowered
with the ability to own their personal data, to control how are been collected, used, shared and by whom.

Nevertheless, the ability to process large volumes of personal information has still resulted in some benefits for individuals and society at large. Big data has positive implications for urban planners who want to design smarter cities, health-care professionals who want to predict epidemics and cure diseases, and engineers who want to identify or even predict new problems to solve.

\section{Regulations}\label{problem:regulations}

The EU Data Protection Directive 95/46/EC (DPR)~\cite{eu-46ec-1995} defines personal data as follows:
\blockquote{
personal data shall mean any information relating to an identified or identifiable natural person ('data subject'); an identifiable person is one who can be identified, directly or indirectly, in particular by reference to an identification number or to one or more factors specific to his physical, physiological, mental, economic, cultural or social identity;
}
However there has been a clear notion that the data subject can potentially identified by pseudo-identifiers~\cite{wiki:pii}.
In the new General Data Protection Regulation (GDPR)~\cite{gdpr} by EU this has been formalised as:
\blockquote{
a data subject is one who can be identified, directly or indirectly, by means reasonably likely to be used by the controller or by any other natural or legal person
}

The recent approval of GDPR~\cite{gdpr} in 2016 by the European Commission (EC)
imposes new obligations on data controllers and processors in contrast to the previously adopted Data Protection Directive (DPR)~\cite{eu-46ec-1995}.
New legislation aims in the extension of responsibility and accountability requirements of organizations also demanding explicit
consent of the data subjects (person) securing their right to withdraw, and to be forgotten~\cite{DBLP:journals/corr/NeisseSF17}.

The key changes of the new data protection law introduced by the GDPR in contrast to the DPR is are~\cite{DBLP:journals/corr/NeisseSF17}:
\begin{itemize}
    \item Organisations based outside EU that process personal data of EU residents are applicable to the new legislation
    \item All EU member states are obliged to a single set of rules
    \item The responsibility and accountability requirements of organisations are extending
    \item EU residents are empowered with explicit consent over their data with the right to be forgotten
\end{itemize}

Data controllers and processors is required to demonstrate to have implemented all the technical and organizational measures needed to ensure the security of the data and the fulfillment of all data protection fundamental principles.

The GDPR increase self-responsibility, and therefore accountability, with particular reference to the processing of sensitive data, such as health and medical data. Privacy lie at the very center of all the new legal and technical paradigms of compliance under the Regulation.

So, data controllers are required to implement all measures and safeguards to ensure that any product or service they provide in is fully in line with the key principles laid down by the GDPR imposing also the obligation to not collect unessential personal data.

Finally, to empower the data subject and have total control over her data, consent rules and a number of fundamental rights, such as the right to be forgotten, have been further consolidated, reinforced or introduced in the GDPR. Therefore, the data subject's will becomes a priority and impose greater attention by the data controllers.

\section{OpenData}\label{problem:opendata}

\section{Motivations for applications}\label{problem:motivations}



banking, healthcare, national security, citizenship documentation or online retailing
