%!TEX root = ../thesis.tex
\chapter{Blockchain Technologies}
\label{blockchain_technologies}

A blockchain is a distributed transaction ledger~\cite{nakamoto2012bitcoin}.
Blockchain consist of a continuously growing list of records, which are called blocks, and are linked, secured and immutable.
Each block typically contains a hash pointer being a link to a previous block, a timestamp and transaction data~\cite{wiki:blockchain}.
Blockchain has a high Byzantine fault tolerance because of a decentralized consensus mechanism.
This makes blockchains potentially suitable for the recording of events, medical records~\cite{blockchain_ehr,Azaria2016}, and other records management activities,
such as identity management, transaction processing or documenting provenance~\cite{wiki:blockchain}. Blockchain could be seen as a distributed immutable, tamper-proof,
and audit log that records every data transaction. As a result any attempt to tamper blockchain is immediately evident and easily detectable.
The first successful financial implementation of blockchain is Bitcoin.
Bitcoin is the first decentralized digital currency and payment system invented by an unknown person or group of people under the name Satoshi Nakamoto~\cite{nakamoto2012bitcoin,wiki:bitcoin}.

\section{Blockchain Types}\label{blockchain_types}

There are various types of blockchains varying in restrictions on data access and participation in the consensus process.
Each one has its own advantages and disadvantages.

\begin{itemize}
  \item Public Blockchain: A public blockchain is a blockchain, in which there are no restrictions on reading blockchain data -encrypted or not- and validating transactions~\cite{prbc_vs_pubbc}.
  The most common implementation of a public blockchain is Bitcoin~\cite{nakamoto2012bitcoin} and Ethereum~\cite{ethash}.
  \item Federated or Consortium blockchain: In a federated blockchain transaction validation is limited to a predefined list of entities with their identities known to the network. Data access can either be public or restricted~\cite{prbc_vs_pubbc}.
  \item Private blockchain: A private blockchain is a blockchain where consensus mechanism is centralized to one single entity regardless of data access~\cite{prbc_vs_pubbc}.
\end{itemize}
