%!TEX root = ../thesis.tex
% the abstract

\abstractEn{
  Data are gathered constantly, grow exponentially, and are considered a
  valuable asset. The need for extensive analysis has emerged by various
  organizations and researchers. However, they can be sensitive, private,
  and protected by privacy disclosure acts making data processing by
  third-parties almost impossible. We propose a protocol for data processing
  where data controllers can register their datasets and entities can
  request data processing operations by data processors. A distributed
  ledger  is used as the controller of the system serving as an immutable
  history log of all actions taken by the participants. The blockchain-based
  distributed ledger provides data accountability, auditability and
  provenance tracking. We also use a Zero Knowledge Verifiable Computation
  scheme where a data processor is enforced to produce a proof of
  correctness of computation without revealing the dataset itself that the
  requestor verifies. This records the fact that correct processing has
  taken place without disclosing any information about the data.
}



\abstractGr{
  Ο όγκος των δεδομένων που συλλέγονται καθημερινά σημειώνει εκθετική αύξηση, ενώ η κατοχή τους θεωρείται πολύτιμη. Η ανάγκη για εκτένη ανάλυση έχει αναδειχθεί μέσα από το έργο διαφόρων ερευνητών και οργανισμών. Ωστόστο, τα δεδομένα αυτά μπορεί να είναι ευαίσθητα και να υπάγονται σε ρυθμιστικές νομοθεσίες απορρήτου κάνοντας την επεξεργασία από τρίτους αδύνατη. Προτείνουμε ένα πρωτόκολλο στο οποίο επεξεργαστές δεδομένων (data processors) έχουν την δυνατότητα να καταχωρήσουν σύνολα δεδομένων (datasets) για τα οποία μπορούν να γίνουν αιτήσεις επεξεργασίας οι οποίες διεκπαιρεώνονται από επεξεργαστές δεδομένων (data processors). Ένα κατανεμημένο μητρώo (distributed ledger) χρησιμοποιείται ως διαχειριστής του συστήματος λειτουργώντας ως ένα αμμετάβλητο ιστορικό όλων των ενεργειών των συμμετεχόντων. Το κατανεμημένο μητρώο παρέχει τις ιδιότητες της λογοδοσίας, του ελέγχου και της παρακολούθησης της προέλευσης των δεδομένων. Επίσης, χρησιμοποιείται ένα σχήμα Μηδενικής Γνώσης Ορθότητας Υπολογισμού (Zero Knowledge Verifiable Computation) μέσα από το οποίο οι επεξεργαστές δεδομένων υποχρεούνται να παράξουν μια απόδειξη ορθότητας υπολογισμού, χωρίς να αποκαλείψουν το ίδιο το σύνολο δεδομένων, την οποία ο αιτών (data requestor) και επαληθεύει. Κατά αυτό τον τρόπο πιστοποιείται το γεγονός ότι πραγματοποιήθηκε η σωστή επεξεργασία δεδομένων χωρίς να αποκαλυφθούν επιπλέον πληροφορίες σχετικά με αυτά.
}
